%%\documentclass[aip,jcp]{revtex4-1}
\documentclass[aip
, pra
, showpacs
, aps
, twocolumn
%, onecolumn
, groupedaddress
, floatfix
%, preprint
]{revtex4}
%]{revtex4-1}
\usepackage{graphicx, amsbsy, bm, dcolumn, amsmath}
\usepackage{currfile}
\usepackage{datetime}

\newcommand{\etal}{{\em et~al.\/}}
\newcommand{\beq}{\begin{equation}}
\newcommand{\eeq}{\end{equation}}
\newcommand{\barr}{\begin{array}}
\newcommand{\earr}{\end{array}}
\newcommand{\JM}{\mbox{\tiny{JM}}}
\newcommand{\Kato}{\mbox{\tiny{Kato}}}
\newcommand{\vare}{\varepsilon}
\newcommand{\phiH}{\phi^{\mbox{\tiny{H}}}}
\newcommand{\vecCT}{ {\bf c} \cdot }
%\newcommand{\vecCT}{ {\bf c}^{\mbox{\tiny{T}} }}



\newcommand{\isum}%
{\mathop{\hbox{$\displaystyle\sum\kern-13.2pt\int\kern1.5pt$}}}

\begin{document}

\title {$J$-matrix calculation of electron-hydrogen $S$-wave scattering: Accurate ionization amplitudes}

\author{Dmitry A. Konovalov}
\affiliation{ARC Centre for Antimatter-Matter Studies}
\affiliation{Discipline of Information Technology, School of Business,
James Cook University, Townsville, Queensland 4811, Australia}

\author{Igor Bray}
\affiliation{ARC Centre for Antimatter-Matter Studies,
Curtin University, GPO Box U1987, Perth, Western Australia 6845, Australia}


\date{\today, \currenttime}

\begin{abstract}
It is shown that existing JM method could be significantly simplified without the loss of convergence rate or accuracy.


The solution is very elegant and arguably somewhat almost trivial, especially when compared to
the existing theories.

\end{abstract}

\pacs{34.80.Dp} %34.80.Dp	Atomic excitation and ionization
\maketitle
The current file is: \currfilename.

\section{INTRODUCTION}
TODO
this Eigenstate Expansion for Scattering (EES) method relies on the following conjecture.

<ultichannel scattering \cite{CA73, NO72, TF79, Nesbet78, Lucchese86}

\section{THEORY}
\subsection{Multichannel scattering}
Since the results of this study were limited to the $S$-wave model of scattering,
only the spin-related quantum numbers are considered in the
following multichannel formalism.

Let $\psi_{0k}$ and $\psi_{1k}$ be regular and irregular solutions of
\beq \barr{l}
(\hat{K}-E)  \psi_{\epsilon k} (r) =0, \ \ \ \epsilon=\{0,1\},\\
\psi_{0k}(r=0) = 0, \ \ \ \psi_{1k}(r=0) \neq 0,\\
\earr \label{H_0_E_psi}\eeq
which are the $S$-terms of the plain wave
\beq
\psi_{0k}(r) = \sin(kr) ,\ \ \
\psi_{1k}(r) = \cos(kr).
\label{K} \eeq
Let $\{\xi_p(r)\}_{p=0}^\infty$ be an $L^2$-complete set of real orthonormal functions
such that for any $p,p'=0,1,...,\infty$
\beq
\langle \xi_p | \xi_{p'} \rangle \equiv \int_0^\infty dr \xi_p(r) \xi_{p'}(r) =\delta_{pp'}, \ \ \ \xi_p(0)=0,
\eeq
\beq
\hat{P}_{N_t} = \sum_{p=0}^{N_t-1} | \xi_p \rangle \langle \xi_p |,
\label{P_N} \eeq


Let the {\em target} Hamiltonian $H_t$ of
$\mathcal{N}$ electrons
is diagonalized within $\hat{P}_N$ (\ref{P_N})
\beq \barr{l}
\langle \Phi_\gamma|H_t|\Phi_{\gamma'}\rangle=e_\gamma \delta_{\gamma\gamma'}, \ \
\langle \Phi_\gamma|\Phi_{\gamma'}\rangle=\delta_{\gamma\gamma'},\\
\Phi_\gamma = \sum_{\beta} C^t_{\gamma \beta} \theta_\beta,
\earr \label{Phi_H_Phi_2} \eeq
where $\{\theta_\beta\}$ denotes a complete set of target electron configurations which could be built from
the first $N$ radial basis functions $\{\xi_p(r)\}_{p=0}^{N-1}$.
In the case of one-electron targets, each
$\theta_\beta$ is just a relevant $\xi_\beta(r)$ together with a required spin component.
For targets with more than one electrons, $\{\theta_\beta\}$ could be constructed as spin-coupled Fano's subshells \cite{Fano65,KFB11},
where $\{\xi_p(r)\}_{p=0}^{N-1}$ are used to define the radial components of the considered subshells.
The total scattering system's eigenstates are constructed identically to the target (\ref{Phi_H_Phi_2})
but for $(\mathcal{N}+1)$-electrons,
\beq \barr{l}
\langle\Psi_i^\Gamma|H|\Psi_j^\Gamma\rangle=E_i\delta_{ij}, \ \
 \langle\Psi_i^\Gamma|\Psi_j^\Gamma\rangle=\delta_{ij},\\
\Psi^\Gamma_i = \sum_{j} C_{ij} \Theta^\Gamma_j,
\earr \label{Psi_H_Psi_2} \eeq
where $H$ is the system Hamiltonian of $(\mathcal{N}+1)$-electrons, $\{\Theta_j^\Gamma\}$
denotes a complete set of system electron configurations, and where
$\Gamma$ denotes quantum numbers conserved by $H$.
For the considered $S$-wave scattering,
$\Gamma=\{S_\Gamma,\mu_\Gamma\}$, where $S_\Gamma$ and $\mu_\Gamma$ are the total spin and its $z$ projection, respectively.


Closely following the JM method \cite{BR76p1491, KFB11} the scattering channels $\{\Phi_\gamma^\Gamma\}$ are defined by spin coupling the target eigenstates
$\{\Phi_\gamma\}$ (\ref{Phi_H_Phi_2})
and the spin of the scattering electron
$s\equiv \frac{1}{2}$
\beq
| \Phi_{\gamma}^{\Gamma} \rangle = \sum_{\mu \mu'}
C_{S_\gamma \mu' s \mu}^{S_\Gamma \mu_\Gamma}
|\Phi_{\gamma \mu'} \rangle \ |s \mu  \rangle,
\label{chi_phi_Gamma} \eeq
where $|s \mu \rangle$ is Pauli spinor,
$C_{j_1m_1j_2m_2}^{jm} \equiv \langle j_1m_1 j_2 m_2| jm\rangle$ are the Clebsch-Gordan coefficients,
and where each scattering channel is labeled by a $\{\gamma, \Gamma\}$-pair.
Then, the solution of the $(\mathcal{N}+1)$-electron Schr\"odinger equation for a given $\Gamma$,
\beq
(H-E) | \Psi_E \rangle =0,  \ \ \  \label{H_E_Psi_E_2}
\eeq
is approximated by
\beq
 \Psi_{E}^\gamma  =  \sum_j a_j^\gamma \Psi_j  
+ \sum_{\alpha}  \sum_{p=N}^\infty \Psi_{\alpha p} f_p^{\alpha \gamma},
\eeq
\beq
\chi_{\epsilon \gamma} = \hat{A} \Phi^\Gamma_{\gamma}  \phi_{\epsilon \gamma},
\eeq
where $\hat{A}$ denotes antisymmetrization operator,
and where the channels' trial functions $\{ \phi_{\epsilon \gamma} \}$ are defined for the open channels as
\beq
\phi_{\epsilon \gamma}\equiv \phi_{\epsilon k_\gamma}, \ \ \   k_{\gamma}^2 = 2(E_i - e_\gamma) > 0.
\eeq



Kato correction is given by
\beq
K_{\gamma' \gamma}^{\Kato} = K_{\gamma' \gamma}^{\JM} - \Delta_{\gamma' \gamma}
\eeq
\beq
\Delta_{\gamma' \gamma} = \langle \Psi_E^{\gamma'}  |H-E| \Psi_E^\gamma \rangle
\eeq

\subsection{$S$-wave $e$-H scattering}
\beq
E_i = e_\gamma + k^2_\gamma/2,
\eeq
\beq \barr{l}
[M_{\epsilon}]_{\gamma \gamma'} = D_1 + D_2 + (-1)^S D_3,\\
D_1 = \langle \Phi_\gamma \xi_{N}  | h_1 + h_2-E_i |
\Phi_{\gamma'} \phi_{\epsilon \gamma'} \rangle, \\
D_2 = \langle \Phi_\gamma \xi_{N}  | v_{12} |
\Phi_{\gamma'} \phi_{\epsilon \gamma'} \rangle, \\
D_3 = \langle \Phi_\gamma \xi_{N}  | H -E_i |
\phi_{\epsilon \gamma'} \Phi_{\gamma'}  \rangle, \\
H_t = \hat{K}_1 + v_1, \\
H = H_t + \hat{K}_2 + V, \ \ \ V = v_2 + v_{12},\\
\earr \label{Y_B_W} \eeq
\beq
D_1 = \delta_{\gamma \gamma'} [ (e_\gamma -  E_i)
\langle \xi_N | \phi_{\epsilon \gamma} \rangle + \langle \xi_N | h_2|\phi_{\epsilon \gamma} \rangle],
\eeq
\beq
D_3 =  \langle \Phi_\gamma \phi_{\gamma}  | v_{12}|
\phi_{\epsilon' \gamma'} \Phi_{\gamma'}  \rangle ,
\label{eH_ex} \eeq







\section{RESULTS}
TODO
\section{CONCLUSIONS}
TODO


\begin{acknowledgments}
\end{acknowledgments}



\bibliographystyle{apsrev}
\bibliography{..//bibtex//qm_references}
%\bibliography{qm_references}
\end{document}
