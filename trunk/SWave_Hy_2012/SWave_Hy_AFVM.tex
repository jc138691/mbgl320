%%\documentclass[aip,jcp]{revtex4-1}
\documentclass[aip
, pra
, showpacs
, aps
, twocolumn
%, onecolumn
, groupedaddress
, floatfix
%, preprint
]{revtex4}
%]{revtex4-1}
\usepackage{graphicx, amsbsy, bm, dcolumn, amsmath}

\newcommand{\etal}{{\em et~al.\/}}
\newcommand{\beq}{\begin{equation}}
\newcommand{\eeq}{\end{equation}}
\newcommand{\barr}{\begin{array}}
\newcommand{\earr}{\end{array}}
\newcommand{\JM}{\mbox{\tiny{JM}}}
\newcommand{\KB}{\mbox{\tiny{KB}}}
\newcommand{\vare}{\varepsilon}
\newcommand{\phiH}{\phi^{\mbox{\tiny{H}}}}
\newcommand{\vecCT}{ {\bf c} \cdot }
%\newcommand{\vecCT}{ {\bf c}^{\mbox{\tiny{T}} }}



\newcommand{\isum}%
{\mathop{\hbox{$\displaystyle\sum\kern-13.2pt\int\kern1.5pt$}}}

\begin{document}

\title {Calculation of excitation and single-ionization amplitudes using a short-range $L^2$ basis}

\author{Dmitry A. Konovalov}
\affiliation{ARC Centre for Antimatter-Matter Studies}
\affiliation{Discipline of Information Technology, School of Business,
James Cook University, Townsville, Queensland 4811, Australia}

\author{Igor Bray}
\affiliation{ARC Centre for Antimatter-Matter Studies,
Curtin University, GPO Box U1987, Perth, Western Australia 6845, Australia}


\date{\today}

\begin{abstract}
It is shown that existing JM method could be significantly simplified without the loss of convergence rate or accuracy.


The solution is very elegant and arguably somewhat almost trivial, especially when compared to
the existing theories.

\end{abstract}

\pacs{34.80.Dp} %34.80.Dp	Atomic excitation and ionization
\maketitle

\section{INTRODUCTION}
TODO
this Eigenstate Expansion for Scattering (EES) method relies on the following conjecture.

<ultichannel scattering \cite{CA73, NO72, TF79, Nesbet78, Lucchese86}

\section{THEORY}
\subsection{One-channel scattering}


After the partial-wave expansion \cite{Taylor72,N82} and focusing only on the $s$-wave terms (angular momentum $l=0$),
electron scattering on a spherically symmetric short-range potential $V(r)$
could be described by the time-independent radial Schr\"odinger equation
\beq
(H-E) \Psi_E (r) =0,  \ \ \  \label{H_E_Psi_E}
\eeq
\beq
H = \hat{K} + V(r), \ \ \ \hat{K} = -(1/2) d^2/dr^2,
\eeq
where $E=k^2/2$ and $k$ are the incident electron energy and momentum, respectively, and where
atomic units are used hereafter.
Let $\psi_{0k}$ and $\psi_{1k}$ be regular and irregular solutions for $\hat{K}$,
\beq \barr{l}
(\hat{K}-E)  \psi_{\epsilon k} (r) =0, \ \ \ \epsilon=\{0,1\},\\ 
\psi_{0k}(r=0) = 0, \ \ \ \psi_{1k}(r=0) \neq 0,\\
\earr \label{H_0_E_psi}\eeq
\beq \barr{l}
\psi_{0k}(r) = \sin(kr) \sqrt{2/k},\\
\psi_{1k}(r) = \cos(kr) \sqrt{2/k},
\earr \label{K} \eeq
where the normalization $\sqrt{2/k}$ becomes convenient in the multichannel scattering.



In 1967, \citet{Harris67} showed that Eq.~(\ref{H_E_Psi_E}) could be solved
approximately using a square-integrable ($L^2$) basis.
Let $\{\xi_p(r)\}_{p=0}^\infty$ be an $L^2$-complete set of real orthonormal functions
such that for any $p,p'=0,1,...,\infty$
\beq
\langle \xi_p | \xi_{p'} \rangle \equiv \int_0^\infty dr \xi_p(r) \xi_{p'}(r) =\delta_{pp'}, \ \ \ \xi_p(0)=0,
\eeq
of which the first $N$ basis functions are used to diagonalize $H$ arriving at
\beq
\langle \Psi_i |H| \Psi_j \rangle = E_i \delta_{ij} , \ \ \langle \Psi_i | \Psi_j \rangle=\delta_{ij},
\eeq
\beq
\Psi_i(r) = \sum_{p=0}^{N-1} C_{ip} \xi_p(r).
\eeq
Then $\Psi_E(r)$ is approximated at the system's eigenvalues $\{E_i\}_{i=1}^{N}$ by $\Psi^N_{E_i}(r)$
\cite{Harris67}
\beq
\Psi^N_{E_i}(r) = \beta_i \Psi_i(r)  + [\phi_{0k_i}(r)  + K_{0} \phi_{1k_i}(r)],
\label{Psi_EES} \eeq
\beq
\phi_{\epsilon k_i}(r=0) = 0, \ \ \epsilon = \{0,1\},
\eeq
where: $E_i=k_i^2/2$; $\{\phi_{\epsilon k}\}$ are the trial continuum functions closely related to $\{\psi_{\epsilon k}\}$; and where the {\em standing}-wave (equal mix of {\em incoming}- and {\em outgoing}-waves) boundary condition is satisfied
\beq
\Psi_E(r \rightarrow \infty) \sim  \sin(kr+\delta_0) \sqrt{2/k}, \ \ \ \Psi_E(0)=0.   \label{stand_wave}
\eeq
The results of the potential $s$-wave elastic scattering could then be expressed via
phase shift ($\delta_0$), real $K$-matrix ($K_0$), complex unitary $S$-matrix ($S_{0}$) and/or cross section ($\sigma_{0}$)
as follows
\beq \barr{l}
K_{0} = \tan{\delta_0},\\
S_{0}=(1+\mbox{i}K_0)(1-\mbox{i}K_0)^{-1}, \\
\sigma_{0}=\pi k^{-2} |S_{0}-1|^2.
\earr \eeq


Consistent with other variational scattering methods \cite{Nesbet68},
\citet{Harris67} solved Eq.~(\ref{H_E_Psi_E}) approximately by requiring
\beq
\langle\Psi_i|H-E_i|\Psi^N_{E_i}\rangle=0,  \label{EES_ONE_C1}
\eeq
and using the following pair of trial functions $\{\phiH_{0k}, \phiH_{1k} \}$ in
Eq.~(\ref{Psi_EES})
\beq
\phiH_{0k}(r) =  \psi_{0k}(r),\ \
\phiH_{1k}(r) =  [\psi_{1k}(r) - \sqrt{\frac{2}{k}} e^{-\lambda r}],
\label{def_phi_H} \eeq
arriving at $K_0\approx K^{\mbox{\tiny{H}}}_{0}$,
\beq \barr{l}
K^{\mbox{\tiny{H}}}_{0} = - \alpha^{\mbox{\tiny{H}}}_{0i} / \alpha^{\mbox{\tiny{H}}}_{1i}, \ \ \ 
\alpha^{\mbox{\tiny{H}}}_{\epsilon i} = \langle\Psi_i|H-E_i|  \phiH_{\epsilon k_i} \rangle,
\earr \label{K_0_Harris} \eeq
where $\lambda$ is a parameter (typically $\lambda=1$ \cite{Nesbet68}) used to make the irregular solution $\psi_{1k}(r)$ regular
at the origin so that it could be used as a trial function in Eq.~(\ref{Psi_EES}).



Our goal here is to improve upon the Harris' {\em Expansion Approach} \cite{Harris67} to the extent that the Eq.~(\ref{Psi_EES})
converges to the exact solution as $N$ increases.
This is attempted by postulating that
within the functional space of $N$ basis $L^2$ functions  $\{\xi_p(r)\}_{p=0}^{N-1}$,
the eigenstates $\{\Psi_i\}_{i=1}^{N}$ are the best possible solutions of the scattering problem (\ref{H_E_Psi_E})
with the standing-wave boundary condition (\ref{stand_wave}) at the corresponding eigenvalues $\{E_i\}_{i=1}^{N}$.
Accepting the conjecture, it immediately follows that the {\em inner} functional space consisting of the first $N$ basis functions
must be removed from the trial functions $\{\phi,\overline{\phi}\}$ in Eq.~(\ref{Psi_EES}) arriving at
\beq 
\phi_{\epsilon k}(r) = (1-\hat{P}_N) \phiH_{\epsilon k}(r),\ \ \ \epsilon = \{0,1\},
\label{def_phi} \eeq
\beq
\hat{P}_N = \sum_{p=0}^{N-1} | \xi_p \rangle \langle \xi_p |,
\label{P_N} \eeq
\beq 
K_{0} = - \alpha_{0i} / \alpha_{1i}, \ \ \ 
\alpha_{\epsilon i} = \langle\Psi_i|H|  \phi_{\epsilon k_i} \rangle.
\label{K_0_EES} \eeq

NOTE: TODO correct above: \cite{HM69} - used orthogonal continuum trial functions


The conjecture could be verified numerically via the general expression for the $K$-matrix \cite{N82}
\beq
\widetilde{K}_0 = -\langle \psi_{0k_i} | V | \Psi_{E_i} \rangle\\
\label{K_theory}\eeq
and its approximation equivalent
\beq
\widetilde{K}^N_0 = -\langle \psi_{0k_i} | V | \Psi^N_{E_i} \rangle.
\label{K_N_theory}\eeq
If $\Psi_i$ is indeed a $\hat{P}_N$-projection of the correct scattering solution $\Psi^N_{E_i}$,
then its functional-space tail could be used to
determine the scale of its contribution to $\Psi^N_{E_i}$ via
\beq \barr{l}
\langle \xi_{N-1} | \Psi^N_{E_i}\rangle
= a_i \langle \xi_{N-1}| \Psi_i \rangle,\\
\langle \xi_{N-1} | \Psi^N_{E_i}\rangle
 =  \langle \xi_{N-1}| \psi_{0k_i} \rangle
+ K_0 \langle \xi_{N-1}| \psi_{1k_i} \rangle,\\
\earr \label{Psi_N_1_tail} \eeq
where $K_0$ is given by Eq.~(\ref{K_0_EES}).
Here it is assumed that $\xi_{N-1}(r)$ is the least vanishing among $\{\xi_p(r)\}_{p=0}^{N-1}$  at large $r$, which is
a common feature of most "standard" $L^2$ bases \cite{abramowitz},
\beq
|\xi_p(r)| < |\xi_{p'}(r)|, \ \ \ p<p', \ \ \ r \rightarrow \infty.
\eeq
Solving Eq.~(\ref{Psi_N_1_tail}) for $a_i$,
the full scattering solution $\Psi^N_{E_i}$ (\ref{Psi_EES}) could be reconstructed and then applied to $\widetilde{K}^N_0$ (\ref{K_N_theory}).
In the case of the one-channel scattering, the convergence
\beq
\lim_{N \rightarrow \infty} |\widetilde{K}^N_0 - K_0 | = 0,
\label{K_N_0_lim} \eeq
could be verified numerically and found to be
remarkably fast, see the results section.



\subsection{Multichannel scattering}
Since the results of this study were limited to the $S$-wave model of scattering,
only the spin-related quantum numbers are considered in the
following multichannel formalism.


Let the {\em target} Hamiltonian $H_t$ of
$\mathcal{N}$ electrons
is diagonalized within $\hat{P}_N$ (\ref{P_N})
\beq \barr{l}
\langle \Phi_\gamma|H_t|\Phi_{\gamma'}\rangle=e_\gamma \delta_{\gamma\gamma'}, \ \
\langle \Phi_\gamma|\Phi_{\gamma'}\rangle=\delta_{\gamma\gamma'},\\
\Phi_\gamma = \sum_{\beta} C^t_{\gamma \beta} \theta_\beta,
\earr \label{Phi_H_Phi_2} \eeq
where $\{\theta_\beta\}$ denotes a complete set of target electron configurations which could be built from
the first $N$ radial basis functions $\{\xi_p(r)\}_{p=0}^{N-1}$.
In the case of one-electron targets, each
$\theta_\beta$ is just a relevant $\xi_\beta(r)$ together with a required spin component.
For targets with more than one electrons, $\{\theta_\beta\}$ could be constructed as spin-coupled Fano's subshells \cite{Fano65,KFB11},
where $\{\xi_p(r)\}_{p=0}^{N-1}$ are used to define the radial components of the considered subshells.
The total scattering system's eigenstates are constructed identically to the target (\ref{Phi_H_Phi_2})
but for $(\mathcal{N}+1)$-electrons,
\beq \barr{l}
\langle\Psi_i^\Gamma|H|\Psi_j^\Gamma\rangle=E_i\delta_{ij}, \ \
 \langle\Psi_i^\Gamma|\Psi_j^\Gamma\rangle=\delta_{ij},\\
\Psi^\Gamma_i = \sum_{j} C_{ij} \Theta^\Gamma_j,
\earr \label{Psi_H_Psi_2} \eeq
where $H$ is the system Hamiltonian of $(\mathcal{N}+1)$-electrons, $\{\Theta_j^\Gamma\}$
denotes a complete set of system electron configurations, and where
$\Gamma$ denotes quantum numbers conserved by $H$.
For the considered $S$-wave scattering,
$\Gamma=\{S_\Gamma,\mu_\Gamma\}$, where $S_\Gamma$ and $\mu_\Gamma$ are the total spin and its $z$ projection, respectively.


Closely following the JM method \cite{BR76p1491, KFB11} the scattering channels $\{\Phi_\gamma^\Gamma\}$ are defined by spin coupling the target eigenstates
$\{\Phi_\gamma\}$ (\ref{Phi_H_Phi_2})
and the spin of the scattering electron
$s\equiv \frac{1}{2}$
\beq
| \Phi_{\gamma}^{\Gamma} \rangle = \sum_{\mu \mu'}
C_{S_\gamma \mu' s \mu}^{S_\Gamma \mu_\Gamma}
|\Phi_{\gamma \mu'} \rangle \ |s \mu  \rangle,
\label{chi_phi_Gamma} \eeq
where $|s \mu \rangle$ is Pauli spinor,
$C_{j_1m_1j_2m_2}^{jm} \equiv \langle j_1m_1 j_2 m_2| jm\rangle$ are the Clebsch-Gordan coefficients,
and where each scattering channel is labeled by a $\{\gamma, \Gamma\}$-pair.
Then, the solution of the $(\mathcal{N}+1)$-electron Schr\"odinger equation for a given $\Gamma$,
\beq
(H-E) | \Psi_E \rangle =0,  \ \ \  \label{H_E_Psi_E_2}
\eeq
is approximated at the system's eigenvalues $\{E_i\}$ (\ref{Psi_H_Psi_2}) by
\beq 
 \Psi_{E_i}  =  \beta_i \Psi_i
+ \sum_{\epsilon \gamma}   \alpha_{\epsilon \gamma}
\hat{A} \Phi^\Gamma_{\gamma}  \phi_{\epsilon \gamma} ,
\eeq
\beq
\chi_{\epsilon \gamma} = \hat{A} \Phi^\Gamma_{\gamma}  \phi_{\epsilon \gamma},
\eeq
\beq
 \Psi_{E_i}  =  \beta_i \Psi_i
+ \vec{\alpha}_0 \cdot \vec{\chi}_0 + \vec{\alpha}_1 \cdot \vec{\chi}_1
\eeq
where $\hat{A}$ denotes antisymmetrization operator,
and where the channels' trial functions $\{ \phi_{\epsilon \gamma} \}$ are defined for the open channels as
\beq
\phi_{\epsilon \gamma}\equiv \phi_{\epsilon k_\gamma}, \ \ \   k_{\gamma}^2 = 2(E_i - e_\gamma) > 0.
\eeq


The unknowns, $\beta_i$ and $\vec{\alpha}_\epsilon$ are determined from two conditions. The first is identical to the one-channel scattering (\ref{EES_ONE_C1})
\beq
\langle\Psi_i|H-E_i|\Psi_{E_i}\rangle=0,
\label{EES_MC_C1} \eeq  %multichannel scattering condition #1
\beq
\langle \vec{\chi}_{0} |H-E_i|\Psi_{E_i}^{\Gamma}\rangle=0,
\label{EES_MC_C2} \eeq  %multichannel scattering condition #2
and yields, in vector notation,
\beq
\vec{B}_0 \vec{\alpha}_0 + \vec{B}_1 \vec{\alpha}_1 = 0, \label{CK_s}
\eeq
\beq
{\bf X}_{00} \vec{\alpha}_0 + {\bf X}_{01} \vec{\alpha}_1 = - \beta_i \vec{B}_0
\eeq
where the vector $\vec{B}$ and matrix ${\bf X}_{\epsilon\epsilon'}$ components are given by
\beq
B_{\epsilon \gamma}
= \langle \Psi_i |H| \chi_{\epsilon \gamma} \rangle,
\eeq
\beq
X_{\epsilon' \gamma', \epsilon \gamma} = \langle \chi_{\epsilon' \gamma'}  | H-E_i |
\chi_{\epsilon \gamma} \rangle,
\label{Y_B_W} \eeq


The solution 
\beq
\vec{\alpha}_1= - {\bf X}_{01}^{-1} \{ \beta_i \vec{B} + {\bf X}_{00} \vec{\alpha}_0 \}
\eeq
\beq \barr{l}
\beta_i = [ \vec{B}_0 - \vec{B}_1 {\bf X}_{01}^{-1} {\bf X}_{00} ] \cdot \vec{\alpha}_0 / x ,\\
x = \vec{B}_1 {\bf X}_{01}^{-1} \vec{B}_0,
\earr \eeq



\subsection{$S$-wave $e$-H scattering}

\beq \barr{l}
\psi_k = \phi_k(r) +  \sum_{p=0}^{N-1} s_{pk} \xi_p(r), \ \ \ s_{pk} = \langle \xi_p | \psi_k \rangle,
\earr \label{Psi_} \eeq


Specifically for the considered $S$-wave $e$-H scattering,
$\Gamma=\{S_\Gamma,\mu_\Gamma\}$, where $S_\Gamma$ and $\mu_\Gamma$ are the total spin and its $z$ projection, respectively.
The relevant Hamiltonians are given by
\beq
H_t = h_1, \ \ \ H = h_1 + h_2 + v_{12},
\label{H_t} \eeq
\beq
h_a = \hat{K}_a  - 1/r_a,\ \ \ v_{12} = 1/\max{(r_{1}, r_{2})}.
\label{h_b} \eeq


Remember definition of a channel: target plus spin of the scattering electron.


\beq \barr{l}
Y_{\gamma'} = \langle \Phi_{\gamma'}  \psi_{\gamma'} |V|\Psi_i^{\Gamma} \rangle,\\
B_{\gamma'} = \langle  \Phi_{\gamma'}  \psi_{\gamma'} |VA^{\Gamma} |\Phi_{\gamma} \phi_{\gamma} \rangle,\\
W_{\gamma' \gamma} = \langle \Phi_{\gamma'}  \psi_{\gamma'} |V A^{\Gamma} |\Phi_{\gamma} \widehat{\phi}_{\gamma} \rangle,\\
\earr \label{Psi_} \eeq

\beq \barr{l}
\langle  \Phi(r_1) \psi(r_2) |f(r_1) g(r_2) - f(r_2) g(r_1) \rangle = \\
=\langle  \Phi |f \rangle  \langle  \psi |g \rangle -
\langle  \Phi |g \rangle  \langle  \psi |f \rangle
\earr \eeq


\section{RESULTS}
TODO: It is shown in the results section that the convergence rate in Eq.~(\ref{K_N_0_lim}) is remarkably fast.

TODO: Figure 1 pot scattering

\section{CONCLUSIONS}
TODO


\begin{acknowledgments}
\end{acknowledgments}



\bibliographystyle{apsrev}
\bibliography{..//bibtex//qm_references}
%\bibliography{qm_references}
\end{document}
