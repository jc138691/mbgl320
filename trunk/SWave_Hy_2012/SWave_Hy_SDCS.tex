%%\documentclass[aip,jcp]{revtex4-1}
\documentclass[aip
, pra
, showpacs
, aps
, twocolumn
%, onecolumn
, groupedaddress
, floatfix
%, preprint
]{revtex4}
%]{revtex4-1}
\usepackage{graphicx, amsbsy, bm, dcolumn, amsmath}

\newcommand{\etal}{{\em et~al.\/}}
\newcommand{\beq}{\begin{equation}}
\newcommand{\eeq}{\end{equation}}
\newcommand{\barr}{\begin{array}}
\newcommand{\earr}{\end{array}}
\newcommand{\vecr}{{\bf r}}
\newcommand{\JM}{\mbox{\tiny{JM}}}
\newcommand{\KB}{\mbox{\tiny{KB}}}
\newcommand{\vare}{\varepsilon}


\newcommand{\isum}%
{\mathop{\hbox{$\displaystyle\sum\kern-13.2pt\int\kern1.5pt$}}}

\begin{document}

\title {Calculation of excitation and single-ionization amplitudes using a short-range $L^2$ basis}

\author{Dmitry A. Konovalov}
\affiliation{Discipline of Information Technology, School of Business, James Cook University, Townsville, Queensland 4811, Australia}

\author{Igor Bray}
\affiliation{ARC Centre for Antimatter-Matter Studies,
Curtin University, GPO Box U1987, Perth, Western Australia 6845, Australia}


\date{\today}

\begin{abstract}
It is shown that existing JM method could be significantly simplified without the loss of convergence rate or accuracy.

\end{abstract}

\pacs{34.80.Dp} %34.80.Dp	Atomic excitation and ionization
\maketitle

\section{INTRODUCTION}
TODO

\section{THEORY}
\subsection{Short-range potential scattering}


Expansion Approach to Scattering by \citet{Harris67}, e-H low energy \cite{MH67}

$(1-\exp(-r))\cos$ \cite{Nesbet68} not correct!



After the partial-wave expansion and retaining only the $s$-wave terms $(l=0)$,
the potential scattering problem is reduced \cite{LL85, Taylor72} to solving the radial Schr\"odinger equation
\beq
(H-E) | \Psi_E \rangle =0, \ \ \ E=k^2/2,  \label{H_E_Psi_E}
\eeq
\beq
\Psi_E(r \rightarrow \infty) \sim \sin(kr+\delta_0), \ \ \ \Psi_E(r= 0)=0,
\eeq
\beq
H=K+V(r), \ \ \ K=-\frac{1}{2} \frac{\mbox{d}^2}{\mbox{d}r^2}, \label{K}
\eeq
where $\delta_0$ is the $s$-wave elastic scattering phase shift and $E$ is the total energy of the system.


Let $\{\xi_p(r)\}_{p=0}^\infty$ be a square-integrable ($L^2$) orthonormal basis
such that
for any $p,p'=0,1,...,\infty$
\beq
\langle \xi_p | \xi_{p'} \rangle=\delta_{pp'}, \ \ \ \xi_p(r=0)=0.
\eeq
The system's Hamiltonian is diagonalized within the functional space of first $N$ basis functions,
\beq
\langle \Psi_i |H| \Psi_j \rangle = E_i \delta_{ij} , \ \ \langle \Psi_i | \Psi_j \rangle=\delta_{ij},
\eeq
\beq
\Psi_i(r) = \sum_{p=0}^{N-1} C_{ip} \xi_p(r).
\eeq
The exact solution $\Psi_E(r)$ is approximated by the sum of {\em inner} and {\em outer} functional space expansions
\beq
\Psi_{E_i}(r) = \Psi_i(r) b_i  + c_0 \hat{\psi}_1(r)  + s_0 \hat{\psi}_2(r), \label{Psi_KB}
\eeq
and where the $s$-wave partial $S$-matrix and elastic cross section are given by
\beq
S_{00}=(1+\mbox{i}R_0)(1-\mbox{i}R_0)^{-1},
\eeq
\beq
\sigma_{00}=\frac{\pi}{k^2} |S_{00}-1|^2.
\eeq


The outer functions $\{\xi_p(r)\}_{p=N}^\infty$ are used to approximate the required boundary condition via
\beq
\hat{\psi}_1(r) c_0 + \hat{\psi}_2(r) s_0 =
(1-\hat{P}_N) \sin(kr + \delta_0),\\
\eeq
where $c_0 = \cos(\delta_0)$, $s_0 = \sin(\delta_0)$,
\beq \barr{l}
\hat{\psi}_1(r) = (1-\hat{P}_N) \sin(kr),\\
\hat{\psi}_2(r) = (1-\hat{P}_N) \cos(kr),
\earr \eeq
and where the inner space projection operator $\hat{P}_N$ is by definition short-range and is given by \\
\beq
\hat{P}_N = \sum_{p=0}^{N-1} | \xi_p \rangle \langle \xi_p |.
\eeq


The unknown expansion coefficients in Eqs.~(\ref{Psi_JM}) and (\ref{Psi_KB})
are found by satisfying the system's  Schr\"odinger Eq.~(\ref{H_E_Psi_E})
separately within the inner and outer functional spaces
arriving at the two set of conditions
\beq 
\langle\Psi_i|H-E_i|\Psi_{E_i}\rangle=0,
\eeq
The first set of conditions are solved for $a_i$ and $b_i$
\beq \barr{l}
S_i c_0 + C_i s_0 = 0, \ \ \
R_0 = - S_i / C_i, \\
S_i = \langle\Psi_i|H-E_i|  \hat{\psi}_1 \rangle, \ \
C_i = \langle\Psi_i|H-E_i|  \hat{\psi}_2 \rangle.
\earr \eeq




One of the main technical issue with the existing JM method is the difficulty of extracting ionization amplitudes.


\subsection{Coulomb-tail potential scattering}



\section{CONCLUSIONS}
TODO


\begin{acknowledgments}
\end{acknowledgments}



\bibliographystyle{apsrev}
\bibliography{..//bibtex//qm_references}
%\bibliography{qm_references}
\end{document}
