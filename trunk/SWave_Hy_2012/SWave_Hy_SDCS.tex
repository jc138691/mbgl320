%%\documentclass[aip,jcp]{revtex4-1}
\documentclass[aip
, pra
, showpacs
, aps
, twocolumn
%, onecolumn
, groupedaddress
, floatfix
%, preprint
]{revtex4}
%]{revtex4-1}
\usepackage{graphicx, amsbsy, bm, dcolumn, amsmath}

\newcommand{\etal}{{\em et~al.\/}}
\newcommand{\beq}{\begin{equation}}
\newcommand{\eeq}{\end{equation}}
\newcommand{\barr}{\begin{array}}
\newcommand{\earr}{\end{array}}
\newcommand{\vecr}{{\bf r}}
\newcommand{\JM}{\mbox{\tiny{JM}}}
\newcommand{\KB}{\mbox{\tiny{KB}}}
\newcommand{\vare}{\varepsilon}


\newcommand{\isum}%
{\mathop{\hbox{$\displaystyle\sum\kern-13.2pt\int\kern1.5pt$}}}

\begin{document}

\title {Calculation of excitation and single-ionization amplitudes using a short-range $L^2$ basis}

\author{Dmitry A. Konovalov}
\affiliation{Discipline of Information Technology, School of Business}
\affiliation{ARC Centre for Antimatter-Matter Studies,
James Cook University, Townsville, Queensland 4811, Australia}

\author{Igor Bray}
\affiliation{ARC Centre for Antimatter-Matter Studies,
Curtin University, GPO Box U1987, Perth, Western Australia 6845, Australia}


\date{\today}

\begin{abstract}
It is shown that existing JM method could be significantly simplified without the loss of convergence rate or accuracy.

\end{abstract}

\pacs{34.80.Dp} %34.80.Dp	Atomic excitation and ionization
\maketitle

\section{INTRODUCTION}
TODO

\section{THEORY}
\subsection{One-channel scattering}


$(1-\exp(-r))\cos$ \cite{Nesbet68} not correct!


After the partial-wave expansion \cite{Taylor72,N82} and focusing only on the $s$-wave terms $(l=0)$,
electron scattering on a spherically symmetric potential $V(r)$
could be described by the time-independent radial Schr\"odinger equation
\beq
(H-E) | \Psi_E \rangle =0,  \ \ \ H = H_0 + V, \label{H_E_Psi_E}
\eeq
where $E=k^2/2$ and $k$ are the incident electron energy and momentum, respectively.
Let $\psi_1$ and $\psi_2$ be real regular and irregular solutions for $H_0$,
\beq \barr{l}
(H_0-E) | \psi_1 \rangle =0,  \ \ \ \psi_1(r=0) = 0,\\
(H_0-E) | \psi_2 \rangle =0,  \ \ \ \psi_2(r=0) \neq 0,\\
\earr \label{H_0_E_psi}\eeq
then for the short-range potential $V(r)$,
\beq \barr{l}
H_0 = \hat{K},\ \ \ \hat{K} = -(1/2) d^2/dr^2,\\
\psi_1(r) = \sin(kr), \ \ \ \psi_2(r) = \cos(kr).
\earr \label{K} \eeq
If present, the long-range Coulomb potential $V_c=-Z/r$ is included into $H_0$
and $\psi_1$ and $\psi_2$ become the corresponding Coulomb wave functions satisfying Eq.~(\ref{H_0_E_psi})
with $H_0 = \hat{K} + V_c$.



In 1967, \citet{Harris67} showed that Eq.~(\ref{H_E_Psi_E}) could be solved
approximately using a square-integrable ($L^2$) basis.
Let $\{\xi_p(r)\}_{p=0}^\infty$ be an $L^2$-complete set of real orthonormal functions
such that for any $p,p'=0,1,...,\infty$
\beq
\langle \xi_p | \xi_{p'} \rangle=\delta_{pp'}, \ \ \ \xi_p(r=0)=0,
\eeq
of which the first $N$ basis functions are used to diagonalize $H$ arriving at
\beq
\langle \Psi_i |H| \Psi_j \rangle = E_i \delta_{ij} , \ \ \langle \Psi_i | \Psi_j \rangle=\delta_{ij},
\eeq
\beq
\Psi_i(r) = \sum_{p=0}^{N-1} C_{ip} \xi_p(r).
\eeq
Then $\Psi_E(r)$ is approximated at the system's eigenvalues $\{E_i\}_{i=1}^{N}$ as
\cite{Harris67}
\beq \barr{l}
\Psi_{E_i}(r) = a_i \Psi_i(r)  + \phi_1(r)  + K_0 \phi_2(r),\\
K_0 = \tan{\delta_0}, 
\earr \label{Psi_EES} \eeq
where the {\em standing}-wave (rather than {\em incoming}- or {\em outgoing}-wave) boundary conditions are satisfied
\beq
\Psi_E(r \rightarrow \infty) \sim \sin(kr+\delta_0), \ \ \ \Psi_E(r= 0)=0,
\eeq
and where $\phi_{\{1,2\}}$ are related to $\psi_{\{1,2\}}$ and will be defined later.
The $s$-wave elastic scattering results could be expressed via:
phase shift ($\delta_0$), $K$-matrix ($K_0$), $S$-matrix ($S_{0}$) and cross section ($\sigma_{0}$)
as follows
\beq \barr{l}
S_{0}=(1+\mbox{i}K_0)(1-\mbox{i}K_0)^{-1}, \\
\sigma_{0}=\pi k^{-2} |S_{0}-1|^2.
\earr \eeq


The goal here is to improve upon the Harris' {\em Expansion Approach} \cite{Harris67} to the extent that the Eq.~(\ref{Psi_EES})
converges to the exact solution as $N$ increases.
To achieve the stated goal, this Eigenstate Expansion for Scattering (EES) method relies on the following conjecture.


{\em Within the functional space of $N$ basis $L^2$ functions  $\{\xi_p(r)\}_{p=0}^N$,
the eigenstates $\{\Psi_i\}_{i=1}^{N}$ are the best possible solutions of the scattering problem in Eq.~(\ref{H_E_Psi_E})
with the standing-wave boundary conditions at the corresponding eigenvalues $\{E_i\}_{i=1}^{N}$.
}


Accepting the conjecture, it immediately follows that the {\em inner} functional space must be removed from
\beq \barr{l}
\hat{\psi}_1(r) = (1-\hat{P}_N) \phi_1(r),\\
\hat{\psi}_2(r) = (1-\hat{P}_N) \phi_2(r),
\earr \eeq
\beq
\hat{P}_N = \sum_{p=0}^{N-1} | \xi_p \rangle \langle \xi_p |.
\eeq

From the general theory of the second-order linear differential equations
\beq \barr{l}
H_0 y(r) = (p(r) y')' - q(r) y,\\
p \equiv -1/2, \ \ q = 0,\\
w = \psi_1(r) \psi_2'(r) - \psi_2(r) \psi_1'(r) \equiv -k,\\
b = (pw)^{-1}= 2/k,
\earr \eeq
TODO: is it the same for Coulomb?
\beq \barr{l}
(H_0 - E) G_0(r,r') = \delta(r-r'),\\
G_0(r,r') = G_0(r',r) = b \psi_1(r_{<}) \psi_2(r_{>}), 
\earr \eeq
\beq \barr{l}
| \Psi_E \rangle = | \psi_1 \rangle + G_0 V |\Psi_E \rangle,\\
\Psi_E(r) = \psi_1(r) +  \int_0^\infty dr'\ G_0(r,r') V(r') \Psi_E(r'),
\earr \eeq
\beq \barr{l}
\Psi_E(r) = \psi_1(r) +  g_2(r) \psi_1(r) + g_1(r) \psi_2(r),\\
\Psi_{E_i}(r) = a_i \Psi_i(r)  + (1-P)\psi_1(r)  \\
\ \ \ + K_0 (1-P)g(r)\psi_2(r),
\earr \eeq

\beq \barr{l}
g(r) \approx b \int_0^r dr'\ \psi_1(r') V(r') \psi_1(r'), \\
g(r) \approx b \int_0^r dr'\ \psi_1(r') V(r') \Psi_i(r'), \\
\earr \eeq

\beq \barr{l}
g_1(r) = b \int_0^r dr'\ \psi_1(r') V(r') \Psi_E(r'), \\
g_2(r) = b \int_r^\infty dr'\ \psi_2(r') V(r') \Psi_E(r'),
\earr \eeq
\beq \barr{l}
K_0 = b \langle \psi_1 | V | \Psi_E \rangle =\\
\ \ = b \int_0^\infty  dr\ \psi_1(r) V(r) \Psi_E(r),
\earr \eeq




The unknown expansion coefficients in Eqs.~(\ref{Psi_JM}) and (\ref{Psi_KB})
are found by satisfying the system's  Schr\"odinger Eq.~(\ref{H_E_Psi_E})
separately within the inner and outer functional spaces
arriving at the two set of conditions
\beq
\langle\Psi_i|H-E_i|\Psi_{E_i}\rangle=0,
\eeq
The first set of conditions are solved for $a_i$ and $b_i$
\beq \barr{l}
S_i c_0 + C_i s_0 = 0, \ \ \
R_0 = - S_i / C_i, \\
S_i = \langle\Psi_i|H-E_i|  \hat{\psi}_1 \rangle, \ \
C_i = \langle\Psi_i|H-E_i|  \hat{\psi}_2 \rangle.
\earr \eeq






\subsection{Coulomb-tail potential scattering}



\section{CONCLUSIONS}
TODO


\begin{acknowledgments}
\end{acknowledgments}



\bibliographystyle{apsrev}
\bibliography{..//bibtex//qm_references}
%\bibliography{qm_references}
\end{document}
