%%\documentclass[aip,jcp]{revtex4-1}
\documentclass[aip
, pra
, showpacs
, aps
, twocolumn
%, onecolumn
, groupedaddress
, floatfix
%, preprint
]{revtex4}
%]{revtex4-1}
\usepackage{graphicx, amsbsy, bm, dcolumn, amsmath}

\newcommand{\etal}{{\em et~al.\/}}
\newcommand{\beq}{\begin{equation}}
\newcommand{\eeq}{\end{equation}}
\newcommand{\barr}{\begin{array}}
\newcommand{\earr}{\end{array}}
\newcommand{\JM}{\mbox{\tiny{JM}}}
\newcommand{\KB}{\mbox{\tiny{KB}}}
\newcommand{\vare}{\varepsilon}
\newcommand{\phiH}{\phi^{\mbox{\tiny{H}}}}


\newcommand{\isum}%
{\mathop{\hbox{$\displaystyle\sum\kern-13.2pt\int\kern1.5pt$}}}

\begin{document}

\title {Calculation of excitation and single-ionization amplitudes using a short-range $L^2$ basis}

\author{Dmitry A. Konovalov}
\affiliation{ARC Centre for Antimatter-Matter Studies}
\affiliation{Discipline of Information Technology, School of Business,
James Cook University, Townsville, Queensland 4811, Australia}

\author{Igor Bray}
\affiliation{ARC Centre for Antimatter-Matter Studies,
Curtin University, GPO Box U1987, Perth, Western Australia 6845, Australia}


\date{\today}

\begin{abstract}
It is shown that existing JM method could be significantly simplified without the loss of convergence rate or accuracy.

\end{abstract}

\pacs{34.80.Dp} %34.80.Dp	Atomic excitation and ionization
\maketitle

\section{INTRODUCTION}
TODO
this Eigenstate Expansion for Scattering (EES) method relies on the following conjecture.

<ultichannel scattering \cite{CA73, NO72, TF79, Nesbet78, Lucchese86}

\section{THEORY}
\subsection{One-channel scattering}


After the partial-wave expansion \cite{Taylor72,N82} and focusing only on the $s$-wave terms (angular momentum $l=0$),
electron scattering on a spherically symmetric potential $V(r)$
could be described by the time-independent radial Schr\"odinger equation
\beq
(H-E) \Psi_E (r) =0,  \ \ \  \label{H_E_Psi_E}
\eeq
\beq
H = \hat{K} + V(r), \ \ \ \hat{K} = -(1/2) d^2/dr^2,
\eeq
where $E=k^2/2$ and $k$ are the incident electron energy and momentum, respectively, and where boundary conditions will be discussed later.
Let $\psi_k$ and $\overline{\psi}_k$ be real (not complex) regular and irregular solutions for $\hat{K}$,
\beq \barr{l}
(\hat{K}-E)  \psi_k (r) =0,  \ \ \ \psi_k(r=0) = 0,\\
(\hat{K}-E)  \overline{\psi}_k (r) =0,  \ \ \ \overline{\psi}_k(r=0) \neq 0,\\
\earr \label{H_0_E_psi}\eeq
\beq \barr{l}
\psi_k(r) = \sin(kr) \sqrt{2/k},\\
\overline{\psi}_k(r) = \cos(kr) \sqrt{2/k},
\earr \label{K} \eeq
where the normalization $\sqrt{2/k}$ will become convenient in the multichannel scattering.



In 1967, \citet{Harris67} showed that Eq.~(\ref{H_E_Psi_E}) could be solved
approximately using a square-integrable ($L^2$) basis.
Let $\{\xi_p(r)\}_{p=0}^\infty$ be an $L^2$-complete set of real orthonormal functions
such that for any $p,p'=0,1,...,\infty$
\beq
\langle \xi_p | \xi_{p'} \rangle=\delta_{pp'}, \ \ \ \xi_p(r=0)=0,
\eeq
of which the first $N$ basis functions are used to diagonalize $H$ arriving at
\beq
\langle \Psi_i |H| \Psi_j \rangle = E_i \delta_{ij} , \ \ \langle \Psi_i | \Psi_j \rangle=\delta_{ij},
\eeq
\beq
\Psi_i(r) = \sum_{p=0}^{N-1} C_{ip} \xi_p(r).
\eeq
Then $\Psi_E(r)$ is approximated at the system's eigenvalues $\{E_i\}_{i=1}^{N}$ by $\Psi^N_{E_i}(r)$
\cite{Harris67}
\beq
\Psi^N_{E_i}(r) = a_i \Psi_i(r)  + [\phi_{k_i}(r)  + K_{0} \overline{\phi}_{k_i}(r)],
\label{Psi_EES} \eeq
where $E_i=k_i^2/2$, the {\em standing}-wave (equal mix of {\em incoming}- and {\em outgoing}-waves) boundary condition is satisfied
\beq
\Psi_E(r \rightarrow \infty) \sim  \sin(kr+\delta_0) \sqrt{2/k}, \ \ \ \Psi_E(r= 0)=0,   \label{stand_wave}
\eeq
and where the trial functions $\{\phi,\overline{\phi}\}$ are related to $\{\psi,\overline{\psi}\}$ and will be discussed shortly.
The results of the potential $s$-wave elastic scattering could then be expressed via
phase shift ($\delta_0$), real $K$-matrix ($K_0$), complex unitary $S$-matrix ($S_{0}$) and/or cross section ($\sigma_{0}$)
as follows
\beq \barr{l}
K_{0} = \tan{\delta_0},\\
S_{0}=(1+\mbox{i}K_0)(1-\mbox{i}K_0)^{-1}, \\
\sigma_{0}=\pi k^{-2} |S_{0}-1|^2.
\earr \eeq
\citet{Harris67} solved Eq.~(\ref{H_E_Psi_E}) by using the pair of trial functions $\{\phiH_k, \overline{\phiH_{k}} \}$ in
Eq.~(\ref{Psi_EES})
\beq \barr{l}
\phiH_k(r) =  \psi_k(r),\ \
\overline{\phiH_k}(r) =  [\overline{\psi}_k(r) - \exp{(-\lambda r)}],
\earr \label{def_phi_H} \eeq
\beq \barr{l}
\langle\Psi_i|H-E_i|\Psi^N_{E_i}\rangle=0,\ \ \
K_{0} = - s / c, \\
s = \langle\Psi_i|H-E_i|  \phiH_{k_i} \rangle, \ \ \
c = \langle\Psi_i|H-E_i|  \overline{\phiH_{k_i}} \rangle,
\earr \label{K_0_Harris} \eeq
where $\lambda$ is a parameter (typically $\lambda=1$ \cite{Nesbet68}) used to make the irregular solution $\overline{\psi}_k(r)$ regular
at the origin so that it could be used as a trial function in Eq.~(\ref{Psi_EES}).



The goal here is to improve upon the Harris' {\em Expansion Approach} \cite{Harris67} to the extent that the Eq.~(\ref{Psi_EES})
converges to the exact solution as $N$ increases.
This is attempted by postulating that
within the functional space of $N$ basis $L^2$ functions  $\{\xi_p(r)\}_{p=0}^{N-1}$,
the eigenstates $\{\Psi_i\}_{i=1}^{N}$ are the best possible solutions of the scattering problem (\ref{H_E_Psi_E})
with the standing-wave boundary condition (\ref{stand_wave}) at the corresponding eigenvalues $\{E_i\}_{i=1}^{N}$.
Accepting the conjecture, it immediately follows that the {\em inner} functional space consisting of the first $N$ basis functions
must be removed from the trial functions $\{\phi,\overline{\phi}\}$ in Eq.~(\ref{Psi_EES}) arriving at
\beq \barr{l}
\phi_k(r) = (1-\hat{P}_N) \psi_k(r),\\
\overline{\phi}_k(r) = (1-\hat{P}_N) [\overline{\psi}_k(r) - \exp{(-\lambda r)}],
\earr \label{def_phi} \eeq
\beq
\hat{P}_N = \sum_{p=0}^{N-1} | \xi_p \rangle \langle \xi_p |,
\label{P_N} \eeq
\beq \barr{l}
K_{0} = - s / c, \ \ \
s = \langle\Psi_i|H|  \phi_{k_i} \rangle, \ \ \
c =  \langle\Psi_i|H|  \overline{\phi}_{k_i} \rangle.
\earr \label{K_0_EES} \eeq


Note that in the case of the potential scattering, the trial pairs
$\{\phiH,\overline{\phiH}\}$ and $\{\phi,\overline{\phi}\}$ yield identical numerical values for $K_0$
via Eqs.~(\ref{K_0_Harris}) and (\ref{K_0_EES}), respectively.
This could be viewed as a partial validation of the proposed method,
since it is known for the Harris' method to be consistent with other variational scattering methods \cite{Nesbet68}.


The presented method could also be checked for
internal theoretical consistency via the general expression for the $K$-matrix \cite{N82}
\beq
\widetilde{K}_0 = -\langle \psi_{k_i} | V | \Psi_{E_i} \rangle\\
\label{K_theory}\eeq
and its approximation equivalent
\beq
\widetilde{K}^N_0 = -\langle \psi_{k_i} | V | \Psi^N_{E_i} \rangle.
\label{K_N_theory}\eeq
If $\Psi_i$ is indeed a $\hat{P}_N$-projection of the correct scattering solution $\Psi^N_{E_i}$, 
then its functional-space tail could be used to
determine the scale of its contribution to $\Psi^N_{E_i}$ via
\beq \barr{l}
\langle \xi_{N-1} | \Psi^N_{E_i}\rangle
= a \langle \xi_{N-1}| \Psi_i \rangle,\\
\langle \xi_{N-1} | \Psi^N_{E_i}\rangle
 =  \langle \xi_{N-1}| \psi_{k_i} \rangle
+ K_0 \langle \xi_{N-1}| \overline{\psi}_{k_i} \rangle,\\
\earr \label{Psi_N_1_tail} \eeq
where $K_0$ is given by Eq.~(\ref{K_0_EES}). 
Here it is also assumed that $\xi_{N-1}(r)$ is the least vanishing among $\{\xi_p(r)\}_{p=0}^{N-1}$  at large $r$, which is 
a common feature of most "standard" $L^2$ bases \cite{abramowitz}, 
\beq
|\xi_p(r)| < |\xi_{p'}(r)|, \ \ \ p<p', \ \ \ r \rightarrow \infty. 
\eeq
Solving Eq.~(\ref{Psi_N_1_tail}) for $a_i$,
the full scattering solution $\Psi^N_{E_i}$ could be reconstructed via Eq.~(\ref{Psi_EES}) and then applied to Eq.~(\ref{K_N_theory}) obtaining
$\widetilde{K}^N_0$. Given that $K_0$  converges to the exact value \cite{Nesbet68}
\beq
\lim_{N \rightarrow \infty} K_0 = \widetilde{K}_0,
\label{K_0_lim} \eeq
it is sufficient to verify that
\beq
\lim_{N \rightarrow \infty} |\widetilde{K}^N_0 - K_0 | = 0.
\label{K_N_0_lim} \eeq
It is shown in the results section that the convergence rate in Eq.~(\ref{K_N_0_lim}) is remarkably fast.



\subsection{Multichannel scattering}
Since the results of this study were limited to the $S$-wave model of scattering, 
only the spin-related quantum numbers are considered in the
following multichannel formalism. The extension to non-zero angular momenta is straight forward, see for example \cite{BR76p1491}. 
 

Let the {\em target} Hamiltonian $H_t$ of   
$\mathcal{N}$ electrons
is diagonalized within $\hat{P}_N$ (\ref{P_N})
\beq \barr{l}
\langle \Phi_\gamma|H_t|\Phi_{\gamma'}\rangle=e_\gamma \delta_{\gamma\gamma'}, \ \
\langle \Phi_\gamma|\Phi_{\gamma'}\rangle=\delta_{\gamma\gamma'},\\
\Phi_\gamma = \sum_{\beta} C^t_{\gamma \beta} \theta_\beta,
\earr \label{Phi_H_Phi} \eeq
where $\{\theta_\beta\}$ denotes a complete set of target electron configurations which could be built from
the first $N$ radial basis functions $\{\xi_p(r)\}_{p=0}^{N-1}$,
and where $\beta$ labeling is done in some arbitrary but predetermined order.
In the case of one-electron targets, each
$\theta_\beta$ is just a relevant $\xi_\beta(r)$ together with a required spin component.
For targets with more than one electrons, $\{\theta_\beta\}$ could be constructed as spin-coupled Fano's subshells \cite{Fano65,KFB11},
where $\{\xi_p(r)\}_{p=0}^{N-1}$ are used to define the radial components of the considered subshells.


The total scattering system's eigenstates are constructed by the procedure identical to the one used for the target (\ref{Phi_H_Phi}) 
but now with the $(\mathcal{N}+1)$-electrons,
\beq \barr{l}
\langle\Psi_i^\Gamma|H|\Psi_j^\Gamma\rangle=E_i\delta_{ij}, \ \
 \langle\Psi_i^\Gamma|\Psi_j^\Gamma\rangle=\delta_{ij},\\
\Psi^\Gamma_i = \sum_{j} C_{ij} \Theta^\Gamma_j,
\earr \label{Psi_H_Psi_2} \eeq
where $H$ is the system Hamiltonian of $(\mathcal{N}+1)$-electrons, $\{\Theta_j^\Gamma\}$
denotes a complete set of system electron configurations, and where 
$\Gamma$ denotes quantum numbers conserved by $H$.
For the considered $S$-wave scattering,
$\Gamma=\{S_\Gamma,\mu_\Gamma\}$, where $S_\Gamma$ and $\mu_\Gamma$ are the total spin and its $z$ projection, respectively.


Closely following the JM method \cite{BR76p1491, KFB11} the scattering channels are defined by spin coupling the target eigenstates 
and the spin of the scattering electron
$s\equiv \frac{1}{2}$
\beq 
| \Phi_{\gamma}^{\Gamma} \rangle = \sum_{\mu \mu'}
C_{S_\gamma \mu' s \mu}^{S_\Gamma \mu_\Gamma}
|\Phi_{\gamma \mu'} \rangle \ |s \mu  \rangle, 
\label{chi_phi_Gamma} \eeq
where $|s \mu \rangle$ is Pauli spinor.
For the considered $S$-wave model, each scattering channel is labeled by a $\{\gamma, \Gamma\}$-pair, where
$\Gamma=\{S_\Gamma,\mu_\Gamma\}$.




Similar to the potential scattering, the solution of the $(\mathcal{N}+1)$-electron Schr\"odinger equation,
\beq
(H-E) | \Psi_E \rangle =0,  \ \ \  \label{H_E_Psi_E_2}
\eeq
is approximated at the system's eigenvalues $\{E_i\}$ from Eq.~(\ref{Psi_H_Psi_2})
\beq \barr{l}
 \Psi_{E_i}^{\Gamma}  =  a \Psi_i^{\Gamma}
 + A^{\Gamma} \Phi_{\gamma}  \phi_{\gamma}
+ \sum_{\gamma''} K_{\gamma''} A^{\Gamma} \Phi_{\gamma''}   \widehat{\phi}_{\gamma''} ,
\earr \label{Psi_} \eeq
Is $K_{\gamma''}$ normalized correctly?
\beq
\phi_{\gamma}\equiv \phi_{k_\gamma}, \ \ \ \psi_{\gamma}\equiv \psi_{k_\gamma}, \ \ \   k_{\gamma}^2 = 2(E_i - e_\gamma),
\eeq


\beq \barr{l}
\langle\Psi_i^\Gamma|H-E_i|\Psi_{E_i}^{\Gamma}\rangle=0,\\
\sum_{\gamma''}  C_{\gamma''} K_{\gamma''}  = -s
, \ \ \ ({\bf C}^t {\bf K}) = -s,\\
s = \langle \Psi_i^\Gamma |HA^{\Gamma} |\Phi_{\gamma} \phi_{\gamma} \rangle,\ \
C_{\gamma'} =  \langle \Psi_i^\Gamma |HA^{\Gamma} |\Phi_{\gamma'} \widehat{\phi}_{\gamma'} \rangle,
\earr \label{Psi_} \eeq

\beq \barr{l}
K_{\gamma'} = - \langle \Phi_{\gamma'}  \psi_{\gamma'}
| V | \Psi_{E_i}^{ \Gamma} \rangle,\\
K_{\gamma'} =   - [a Y_{\gamma'}   + B_{\gamma'}
+ \sum_{\gamma''}  W_{\gamma' \gamma''} K_{\gamma''}],\\
{\bf K} =   -[a {\bf Y} + {\bf B} + {\bf W K}],\\
{\bf K}  = -({\bf I}+{\bf W})^{-1}   (a {\bf Y}    + {\bf B}),
\earr \eeq

\beq \barr{l}
s = a y  + b,   \\
a = (s - b) / y,\\
y = {\bf C}^t ({\bf I}+{\bf W})^{-1}  {\bf Y}, \\
b = {\bf C}^t ({\bf I}+{\bf W})^{-1}  {\bf B},
\earr \eeq
\beq \barr{l}
Y_{\gamma'} = \langle \Phi_{\gamma'}  \psi_{\gamma'} |V |\Psi_i^{\Gamma} \rangle,\\
B_{\gamma'} = \langle \Phi_{\gamma'}  \psi_{\gamma'} |V A^{\Gamma} |\Phi_{\gamma} \phi_{\gamma} \rangle,\\
W_{\gamma' \gamma} = \langle \Phi_{\gamma'}  \psi_{\gamma'} |V A^{\Gamma} |\Phi_{\gamma} \widehat{\phi}_{\gamma} \rangle,\\
\earr \label{Psi_} \eeq

\beq \barr{l}
\psi_k = \phi_k(r) +  \sum_{p=0}^{N-1} s_{pk} \xi_p(r), \ \ \ s_{pk} = \langle \xi_p | \psi_k \rangle,
\earr \label{Psi_} \eeq


\subsection{$S$-wave $e$-H scattering}

Specifically for the considered $S$-wave $e$-H scattering,
$\Gamma=\{S_\Gamma,\mu_\Gamma\}$, where $S_\Gamma$ and $\mu_\Gamma$ are the total spin and its $z$ projection, respectively.
The relevant Hamiltonians are given by
\beq
H_t = h_1, \ \ \ H = h_1 + h_2 + v_{12},
\label{H_t} \eeq
\beq
h_a = \hat{K}_a  - 1/r_a,\ \ \ v_{12} = 1/\max{(r_{1}, r_{2})}.
\label{h_b} \eeq


Remember definition of a channel: target plus spin of the scattering electron.


\beq \barr{l}
Y_{\gamma'} = \langle \Phi_{\gamma'}  \psi_{\gamma'} |V|\Psi_i^{\Gamma} \rangle,\\
B_{\gamma'} = \langle  \Phi_{\gamma'}  \psi_{\gamma'} |VA^{\Gamma} |\Phi_{\gamma} \phi_{\gamma} \rangle,\\
W_{\gamma' \gamma} = \langle \Phi_{\gamma'}  \psi_{\gamma'} |V A^{\Gamma} |\Phi_{\gamma} \widehat{\phi}_{\gamma} \rangle,\\
\earr \label{Psi_} \eeq

\beq \barr{l}
\langle  \Phi(r_1) \psi(r_2) |f(r_1) g(r_2) - f(r_2) g(r_1) \rangle = \\
=\langle  \Phi |f \rangle  \langle  \psi |g \rangle -
\langle  \Phi |g \rangle  \langle  \psi |f \rangle
\earr \eeq


\section{RESULTS}
TODO: Figure 1 pot scattering

\section{CONCLUSIONS}
TODO


\begin{acknowledgments}
\end{acknowledgments}



\bibliographystyle{apsrev}
\bibliography{..//bibtex//qm_references}
%\bibliography{qm_references}
\end{document}
