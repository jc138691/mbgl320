\documentclass[11pt]{article}
\setlength{\oddsidemargin}{.25in}
\setlength{\topmargin}{-.25in}
\setlength{\textheight}{8.75in}
\setlength{\textwidth}{6in}
\setlength{\parindent}{.25in}
\usepackage{moreverb}
\usepackage{longtable}
\usepackage{textcomp}
\usepackage{graphicx} 
\usepackage{amstext,amssymb}
\usepackage{pslatex}
\usepackage{url}
\usepackage[ps2pdf,colorlinks]{hyperref}

\begin{document}

\pagenumbering{roman}
\thispagestyle{empty}
\begin{center}
\huge
\bf
User's Guide to f2j \\
Version 0.8
\vspace*{1in} \mbox{} \\
\LARGE \rm
Keith Seymour and Jack Dongarra 

\vspace*{.5in}
Innovative Computing Laboratory\\
Department of Computer Science\\
University of Tennessee\\
\vspace*{.5in}
May 31, 2007
\end{center}


%\include{license}
%\tableofcontents
%\listoftables
%\listoffigures

\newpage
\pagenumbering{arabic}
\setcounter{page}{1}

\section{Introduction}

Before using the f2j source code, realize that f2j was originally geared
to a very specific problem - that is, translating the LAPACK and BLAS numerical
libraries.  However, now that the translation of the single and double precision
versions of BLAS and LAPACK is complete, the goal is
to handle as much Fortran as possible, but there's still a lot left to cover.
We have a lot of confidence in the JLAPACK translation, but for a variety of
reasons, f2j will most likely not correctly translate your code at first.

One of the reasons for putting the code up on SourceForge is to enable easy
collaboration with other developers.  If you're interested in helping the
development of f2j, we'll consider giving commit access to the CVS tree.

The purpose of this document is to describe how to build and use the f2j compiler and
to give some background on how to extend it to handle your Fortran code. 

\section{Obtaining the Code}

For downloads and CVS access, see the f2j project page at SourceForge:

\begin{verbatim}
  http://sourceforge.net/projects/f2j
\end{verbatim}

There is a source tarball available in the download section and anonymous
CVS access is also available.

The GOTO translation code is based on the bytecode parser found in javab, a
bytecode parallelizing tool under development at the University of Indiana.
That code is covered under its original license, found in the translator source
directory.

\section{Limitations}

There are many limitations to be aware of before using f2j:
\begin{itemize}
\item Parsing -- the parser has a bug that requires at least one variable
declaration in every program unit.  Also, the last line of the program
cannot be blank.
\item Typechecking -- f2j does not aim to do much typechecking.  It assumes
that you have already tested the code with a real Fortran compiler.
\item Data types -- complex numbers are not supported.
\item Input/Output -- f2j does not support any file I/O.  Formatted
I/O support is fairly weak, but works for many simple cases.  At worst, the
output will be missing or you'll get ``NULL'' printed out instead of numbers.
\item Other Features -- Certain forms of Fortran EQUIVALENCE are not
supported.  f2j can handle a limited form of EQUIVALENCE as long as the variables
being equivalenced do not differ in type and are not offset from each other.
Multiple entry points are not supported.
\end{itemize}

With that said, if you have pretty straightforward numerical code (similar to
BLAS or LAPACK) f2j may be able to handle it.

\section{Building and Using f2java}

We have been doing development and testing of f2j on Sun SPARCstations running
various versions of Solaris as well as x86 machines running various versions
of Linux and Solaris/x86.  It may compile on
other platforms, though.  Using gcc 3.4.4 with the \verb|-Wall| flag, we get no
warnings, but using some picky compilers, you may see warnings about 
unused variables, etc.  You can safely ignore them.

First, download and uncompress the source code.  Building the code follows the typical configure/make process:

\begin{verbatim}
# ./configure
# make
\end{verbatim}

Optionally, you can ``make install'' which will copy the executables to the
location specified in the \verb|--prefix| argument to configure.

Now you may want to add the relevant install directory to your PATH.  This
will vary depending on whether you did ``make install''.  If so, the PATH
should include \verb|$prefix/bin|.  If not, your PATH should include
\verb|$f2j_dir/src| and \verb|$f2j_dir/goto_trans|, where \verb|$f2j_dir|
is the top-level f2j source directory.  You may also want to modify your 
CLASSPATH to include the f2j util package.
If you did ``make install'', this will be \verb|$prefix/lib/f2jutil.jar|.
Otherwise, it will be \verb|$f2j_dir/util/f2jutil.jar|.

Let's go through a simple example.  Say you have the following Fortran code in a file 
called ``test.f''

\begin{verbatim}
      program blah
      external foo
      write(*,*) 'hi'
      call foo(12)
      stop
      end
      subroutine foo(y)
      integer y
      write(*,*) 'foo ', y
      return
      end
\end{verbatim}

If you translate it with ``\verb|f2java test.f|'', it will produce one class file and
one Java source file for each program unit.  So, in this case since we have 
two program units in the Fortran source file, we end up with four generated files:
Blah.java, Blah.class, Foo.java, and Foo.class (note the first letter
of the name becomes
capitalized).  You can run the generated class file directly:

\begin{verbatim}
# java Blah
hi
foo  12
\end{verbatim}

You don't need to compile the Java source, but if you wanted to modify it,
you could recompile:

\begin{verbatim}
# javac Blah.java Foo.java
\end{verbatim}

However at this point the GOTO statements haven't been converted, so if you
run it you'll see some warnings like this:

\begin{verbatim}
# java Blah
hi
foo  12
Warning: Untransformed goto remaining in program! (Foo, 999999)
Warning: Untransformed label remaining in program! (Foo, 999999)
\end{verbatim}

So you need to run the GOTO transformer (javab) on the class files:

\begin{verbatim}
# javab *.class
\end{verbatim}

and then it'll run fine:

\begin{verbatim}
# java Blah
hi
foo  12
\end{verbatim}

\section{Command-line Options}

There are several command-line options that you should be aware of: 

\begin{itemize}
\item -I specifies a path to be searched for included files (may be used
  multiple times).

\item -c specifies the search path for f2j ``descriptor'' files (ending in .f2j).
  It is a colon-separated list of paths, like a Java CLASSPATH).  For example:
\begin{verbatim}
    f2java -c .:../objects filename.f
\end{verbatim}

\item -p specifies the name of the package.  For example:
\begin{verbatim}
    f2java -p org.netlib.blas filename.f
\end{verbatim}

\item -o specifies the destination directory to which the code should be
  written.

\item -w forces all scalars to be generated as wrapped objects.  The default
  behavior is to only wrap those scalars that must be passed by reference.
  Note that using this option will generate less efficient Java code.

\item -i causes f2j to generate a high-level interface to each subroutine and
  function.  The high-level interface uses a Java-style calling convention (2D
  row-major arrays, etc).  The low-level routine is still generated because the
  high-level interface simply performs some conversions and then calls the
  low-level routine.

\item -h displays help information.

\item -s causes f2j to simplify the interfaces by removing the offset parameter and
  using a zero offset.  It isn't necessary to specify -i in addition to -s.

\item -d causes f2j to generate comments in a format suitable for javadoc.  It is a
  bit of a LAPACK-specific hack -- the longest comment in the program unit is
  placed in the javadoc comment.  It works fine for BLAS/LAPACK code (or any
  other code where the longest comment is the one that describes the function),
  but will most likely not work for other code.

\item -fm causes f2j to generate code that calls java.lang.StrictMath
  instead of java.lang.Math.  By default, java.lang.Math is used.

\item -fs causes f2j to declare the generated code as strictfp (strict
  floating point).  By default, the generated code is not strict.

\item -fb enables both the -fm and -fs options.

\item -vs causes f2j to generate all variables as static class
  variables.  By default f2j generates variables as locals.

\item -va causes f2j to generate arrays as static class variables, but
  other variables are generated as locals.
\end{itemize}

After issuing the command ``f2java file.f'' there should be one or more Java
files in your current directory, one Java file and one class file per Fortran program unit
(function, subroutine, program) in the source file.  Initially, we would
suggest concatenating all Fortran program units into one file because it makes
it easier to perform correct code generation (more about this later).  As the
example above illustrated, you can run the class file corresponding to the main
Fortran program unit or you can use the Java compiler of your choice to compile
the resulting Java source code.  Make sure that the org.netlib.util package is
in your CLASSPATH.  This package comes in both the f2j and JLAPACK distributions,
so if your CLASSPATH already points to JLAPACK's f2jutil.jar, then you're ok.

\section{Organizing Your Fortran Code}

Any non-trivial Fortran program will consist of multiple source files, often
in many different directories.  This can present difficulties for f2j because 
resolving external functions and subroutines is critical for generating the
call correctly.  

First, we will give some practical advice on organizing your code to be built
using f2j.  The following section will give a more detailed explanation of
why this is all so important.

\subsection{Practical Aspects}

The easiest method is to just concatenate all your Fortran code into one file
and run f2j on it.  This might not be practical in all cases, though.  If you have
to keep code in separate files, you need to understand the dependence relationship
between them.  For example, if you have files \verb|a.f| and \verb|b.f|, and
routines in \verb|a.f| call routines in \verb|b.f|, then you must translate
\verb|b.f| first.  If there is a cross dependency, then f2j will most likely
not generate some calls correctly.  Thinking of it as a call tree, you want
to start translating at the leaves and work your way back up.  This sometimes
requires modifying the code.

When code exists in separate subdirectories, the procedure is largely the same,
except that f2j needs to know the subdirectory names containing files that the
current program unit depends on.  Modifying the previous example, let's say
that \verb|b.f| is in a subdirectory named \verb|../code/foo|.  We would first
go to \verb|../code/foo| and translate \verb|b.f|, which would result in the 
creation of a number of descriptor files ending in \verb|.f2j|.  Then in the subdirectory
containing \verb|a.f|, specify the other subdirectory on the command line:
\begin{verbatim}
# f2java -c .:../code/foo a.f
\end{verbatim}
f2j will locate the descriptor files in \verb|../code/foo| and use them to
generate the correct calls to the routines contained in \verb|b.f|.  You can
specify multiple paths separated by a colon.

\subsection{Resolving External Routines}
This section illustrates in more detail the importance of resolving calls to functions
or subroutines which do not appear in the original source file.
By ``resolving'', we mean determining the correct
calling sequence for the function call, which
depends on its method signature.  For example, consider
the following Fortran program segment:
\begin{verbatim}
      INTEGER X(10)

      CALL FUNC1( X(5) )
      CALL FUNC2( X(5) )
 [...]
      SUBROUTINE FUNC1(A)
      INTEGER A
 [...]
      SUBROUTINE FUNC2(A)
      INTEGER A(*)
\end{verbatim}

\begin{table*}[t]
\begin{center}
\begin{sffamily}
  \begin{tabular}{ll}
  \hline
  \textbf{Calling FUNC1} & \textbf{Calling FUNC2}\\ \hline
 \verb|getstatic #15 <Field Hello.x:int[]>|
 &  \verb|getstatic #15 <Field Hello.x:int[]>| \\

 \verb|iconst_5|
 &  \verb|iconst_5|  \\

 \verb|iconst_1|
 &  \verb|iconst_1|  \\

 \verb|isub|
 &  \verb|isub|  \\

 \verb|iaload|
 &  \verb|invokestatic #28|  \\

 \verb|invokestatic #22|
 &  \verb|    <Method Func2.func2(int[],int):void>|  \\

 \verb|    <Method Func1.func1(int):void>|
 &  \verb|| \\

  \hline
  \end{tabular}
\end{sffamily}
\end{center}
\caption{Differences in Argument Passing.}
\label{tab:argpass}
\end{table*}

The first subroutine, {\tt FUNC1}, expects a scalar argument,
while {\tt FUNC2} expects an array argument.
These two calls would be generated identically in a standard
Fortran compiler, regardless of how {\tt FUNC1} and
{\tt FUNC2} were defined --- the address of the fifth element of X
would be passed to the subroutine in both cases.  However, things
are not as simple in Java due to the lack of pointers.
To simulate passing array subsections, as necessary for the
second call, we actually pass two arguments --- the array
reference and an additional integer offset parameter, as shown
in the right column of Table \ref{tab:argpass}.

However, the first subroutine expects a scalar, so we should pass only the value
of the fifth element, without any offset parameter, as shown in the left column
of Table \ref{tab:argpass} (in this case, assume that {\tt FUNC1} does not
modify the argument, otherwise things get even more complex).

Notice that the primary difference between the two calling sequences
is that when calling {\tt FUNC1}, the array is first dereferenced using the {\tt iaload}
instruction.  Also note that the purpose of the arithmetic expression is
to decrement the index by 1 to compensate for the fact that
Java has 0-based indexing whereas Fortran has 1-based indexing.

The only way to determine the correct calling sequence for
any given call is to examine the
parameters of the corresponding subroutine or function declaration.
This is only possible if the declaration had been
parsed at the same time as the current program unit, meaning
that for code generation to work properly all the source files
had to be joined into a big monolithic input file.

This was a serious limitation, especially for large libraries,
because a modification to any part of the code requires
re-compiling {\em all} the source.
There are at least a couple of ways to solve this problem.
One way would be to obtain the parameter information directly
from class files that have already been generated.  While this
would work well, f2j is written in C
and does not have access to nice Java features like reflection,
so it would require a lot of extra code
to parse the class files.
Instead, we use a more lightweight
procedure in f2j.  At compile-time, f2j creates a {\it descriptor file} which
is a text file containing a list of every method generated.  Each
line of the descriptor file contains the following information:
\begin{itemize}
\item Class name -- the fully qualified class name which contains
the given method.
\item Method name -- the name of the method itself.
\item Method descriptor -- this method's descriptor, which is
a string representing the types of all the arguments as well
as the return type.
\end{itemize}
Continuing with the previous example, the descriptor files 
for {\tt FUNC1} and {\tt FUNC2} would be:
\begin{verbatim}
# cat Func1.f2j
Func1:func1:(I)V

# cat Func2.f2j
Func2:func2:([II)V
\end{verbatim}

To resolve a subroutine or function call, we search all
the descriptor files for the matching method name and examine
the method descriptor.  Based on the method descriptor, we can then
correctly generate the calling sequence.  
The code generator
locates the descriptor files based on colon-separated paths specified
on the command line or
in the environment variable {\tt F2J\_SEARCH\_PATH}.

\section{Extending f2j}

So, at this point you may be wondering how to extend f2j to handle your code.
Typically, the first problem you'll run into is that f2j doesn't parse your
code. That could involve something as simple as changing a production in the
parser or it could involve a bit more work - e.g. creating a new kind of AST
node along with all the appropriate code generation routines. The first thing
you'll want to check is whether the parser supports the syntax your code uses
(the parsing code is machine generated from a Yacc grammar in f2jparse.y). For
example, if your code contains an ENTRY statement, your code will not compile
because f2j doesn't support alternate entry points.  Suppose you wanted to
implement ENTRY in f2j. Your first step would be to define a lexer token to
represent the ENTRY keyword (in fact, this exists already, even though ENTRY is
not implemented).  The lexer sometimes needs to be modified to handle the token
correctly, but usually it is sufficient to put the token in the appropriate
lexer table.  In this case, we would just put the ENTRY keyword in the
\verb|tab_stmt| array defined in \verb|globals.c|.  That array holds keywords
that are at the beginning of statements.  You'll notice that this has also been
added already.

If you're getting parse errors on a line of code that should compile based on
your examination of the parser, then the lexer might not be sending the correct
tokens to the parser. The lexical analysis code is in f2jlex.c, which is
handwritten C code based on Sale's algorithm. There's not really an easy way of
describing the structure of the lexer code, but if you enable debugging output
(set lexdebug = TRUE) it will show which tokens are being passed from the lexer
to the parser.  That should help you figure out where the problem is.

While you're working on the parsing, you can leave the code section in the Yacc
grammar blank.  You'll recognize when it finally parses correctly because
you'll get a segmentation fault (meaning it passed the parsing phase and
failed in a subsequent phase since you didn't pass an AST node back up from
that production).  At this point, you need to determine what information is
needed by the back-end to generate the code.  For example, a loop might need
a statement label number, an initial value, a final value, and an increment
value.  The AST node types are defined in \verb|f2j.h|.  If the node you're
defining is close enough to an existing node, you can reuse it.  Otherwise you'll
have to create a new one.  Then just initialize this node in the code
section for your new production.

If f2j can parse your code, but the resulting Java code does not compile or
does not work, then this may indicate a problem in the f2j back-end. First, try
concatenating all your Fortran files into one big file (ok, we admit this is
cheesy, but it does work sometimes). This should help with the type analysis
phase and may eliminate problems in the resulting Java code. After that, if the
generated code is still incorrect, begin looking into the f2j code. After f2j
parses your code, it passes through a couple of stages before actually
generating code.  First, the AST goes through ``type analysis'' (typecheck.c),
which simply means that the tree is fully traversed and each node is assigned
type information as appropriate. This is not semantic analysis, just
annotation. Next, the AST goes through ``scalar optimization'' (optimize.c),
which is an optimization stage designed to determine which scalar variables
need to be wrapped in objects and which can remain primitives. After that, f2j
generates the Java code (codegen.c) based on the modified AST. So, if you
notice a type mismatch problem in the generated code, typecheck.c would be a
good place to begin debugging. Similarly, if you notice that object wrappers
are inappropriately used, check into optimize.c (hint: by passing the -w flag
to f2java, the scalar optimization code will be skipped). Most other problems
will be with the code generator itself.

\end{document}
