%%\documentclass[aip,jcp]{revtex4-1}
\documentclass[aip
, pra
, showpacs
, aps
, twocolumn
%, onecolumn
, groupedaddress
, floatfix
%, preprint
]{revtex4}
%]{revtex4-1}
\usepackage{graphicx, amsbsy, bm, dcolumn, amsmath}

\newcommand{\etal}{{\em et~al.\/}}
\newcommand{\beq}{\begin{equation}}
\newcommand{\eeq}{\end{equation}}
\newcommand{\barr}{\begin{array}}
\newcommand{\earr}{\end{array}}
\newcommand{\vecr}{{\bf r}}
\newcommand{\dd}{\mbox{d}}
\newcommand{\dr}{\mbox{d} r}
\newcommand{\vare}{\varepsilon}
\newcommand{\calN}{{\cal N}}
\newcommand{\di}{_{\mbox{\tiny{di}}}}
\newcommand{\ex}{_{\mbox{\tiny{ex}}}}



\newcommand{\isum}%
{\mathop{\hbox{$\displaystyle\sum\kern-13.2pt\int\kern1.5pt$}}}

\begin{document}

\title {$J$-matrix calculation of electron-helium-ion $S$-wave scattering}

\author{Dmitry A. Konovalov}
\affiliation{Discipline of Information Technology, School of Business}
\affiliation{ARC Centre for Antimatter-Matter Studies, James Cook University, Townsville, Queensland 4811, Australia}

\author{Dmitry V. Fursa}
\affiliation{ARC Centre for Antimatter-Matter Studies,
Curtin University, GPO Box U1987, Perth, Western Australia 6845, Australia}

\author{Igor Bray}
\affiliation{ARC Centre for Antimatter-Matter Studies,
Curtin University, GPO Box U1987, Perth, Western Australia 6845, Australia}



\date{\today}

\begin{abstract}
todo

\end{abstract}

\pacs{34.80.Dp} %34.80.Dp	Atomic excitation and ionization
\maketitle



\section{INTRODUCTION}


The $J$-matrix (JM) method \cite{HY74p1201,BR76p1491} is a  general method for solving wide range of scattering problems.
In this study, we continue to develop a version of the JM method that
was successfully applied to the $S$-wave $e$-H  ($S$-$e$-H) \cite{KB10p022708}  and $S$-wave $e$-He  ($S$-$e$-He) \cite{KFB11,KFB12} scattering problems,
where hereafter this version will be referred to as the KFB method.


This study focuses on the $S$-wave $e$-He$^+$ ($S$-$e$-He$^+$) scattering model \cite{MGB95,BB97,BC97},
where only the partial wave with zero angular momentum ($l=0$) is retained in all calculations
and partial-wave expansions. Our motivation and goal here are twofold. First, the
$S$-$e$-He$^+$ model contains the long-range Coulomb interaction of the scattering electron and the target helium-ion in all scattering channels.
While the general mathematical JM expressions for the Coulomb scattering have been known since 1975 \cite{YF75}, surprisingly,
most (if not all) JM calculations have only been applied to the short-range non-Coulomb scattering problems.
Hence, it is interesting to apply the JM method to the Coulomb electron-ion scattering as it has not been done before.


The second goal of this study is to provide high accuracy $S$-$e$-He$^+$ cross sections by combining advantages of the convergent-close-coupling (CCC) and JM methods.
The JM and CCC methods are implemented independently and use completely different approaches to solve the scattering equations.
Therefore, the CCC and JM methods can be and were used to cross-verify that their results are convergent within their own numerical parameters at key energy points,
where the JM method is then used \cite{HY74p1201,BR76p1491} to calculate a vast number of energy points required for the benchmark results.


See \cite{JMatrixWebsite} for information on availability of the results and source code.



The need for such benchmark calculations is evident from the existing {\em ab initio} attempts to solve the $S$-$e$-He problem.
Reviewing in reverse chronological order, in 2010, \citet{BS10p022715} developed a four-body propagating exterior scaling (PECS) method and reported results claiming to achieve "benchmark" level of accuracy.
However none of their cross sections, including elastic and $2^{1,3}S$ excitation cross sections, displayed resonances at the accuracy level achieved for the $S$-$e$-H problem \cite{P78}.
In 2005, \citet{HMR05} reported results using
time-dependent exterior complex scaling (TD-ECS), which also failed to described resonance behavior of the cross sections.
In 2002 and 2004, the CCC method \cite{PBFS02,PNBFS04} did not examine the resonance regions with sufficiently fine energy grid.
This is now corrected to some extent when in 2011 \citet{KFB11} reported the CCC and JM {\em frozen-core}
(FC) results clearly showing the resonances in the elastic and $n=2$ ($2^1S$ and $2^3S$) excitation cross sections.
The RM method \cite{FLRS94b, SMC2006} has never reported its results for the $S$-$e$-He problem.
In summary, while the three most recent methods (PECS, TD-ECS, CCC) are in good qualitative or overall agreement with each other,
in many specific cases their results differ by as much as 10-20\% \cite{BS10p022715,HMR05}.
These discrepancies together with the failure to describe the resonance features of the model,
is a major theoretical gap as far as the scattering theory is concerned.



The $S$-wave models have proven to be a very productive testing ground for {\it ab initio} scattering theories,
see \cite{T62,HY74p1209,P78,P80,P81,CO84,BS92p53,BST93,KM94pL407,IDHF95,PS96,JS02,JS00l,BRIM99,S99l,MHR02,BS04,Frapiccini10} for the $S$-wave $e$-H scattering ($S$-$e$-H)
and \cite{DHIF94,PMR99,PBFS02,PNBFS04,HMR05R,HMR05,BS10p022715,BS10p022716,KFB11} for  the $S$-$e$-He problem.


The main attraction of the $S$-wave models is that they retain most of the physics complexities of the full
scattering problems while reduce the problems computationally.





\section{THEORY}

\subsection{Radial grid}
Computational efficiency remains an important aspect of most {\em ab initio} scattering methods. 
In particular, very efficient numerical methods are available if grid-points ${\bf x} = (x_1, x_2, ..., x_M)$
are equally spaced with respect to the independent variable $x(r)$, where $r$ is the radial coordinate.
A number of such transformations are available. The logarithm of $r$ (LR) 
\beq
x(r) = \ln(r)
\eeq
is efficient when working with bound state wave functions \cite{F73,FFischer77,FFBJ97}, where the near-zero region of $r$ is not transformed.
The logarithm of constant $c$ plus $r$ (LCR) \cite{KB10p022708}
\beq
x(r) = \ln(c+r)
\eeq
creates a single ${\bf x}$-grid including the near-zero region of $r$. The mixed linear and logarithmic transformation (LLR)
\beq
x(r) = ar + b\ln(r),
\eeq
is preferred over the LR grid when working with highly excited and/or continuum states \cite{CCR76}, 
where $a$ and $b$ are constants, and (as in LR) the near-zero region of $r$ is not transformed. 
While the LCR grid was very efficient when working with the Laguerre basis functions \cite{KB10p022708, KFB11, KFB12}, 
the LCR grid is not suitable for working explicitly with continuum states. 
Using the same arguments as for the LCR grid \cite{KB10p022708}, the LLR grid is modified arriving at the MCR grid 
\beq
x(r) = ar + b\ln(y), \ \ y = c+r,
\eeq
where (as for LCR) the near-zero region of $r$ is explicitly included in the grid, and where MCR mnemonic stands for modified or mixed LCR.


Given such a variety of possible transformations, it seems convenient to give 
a summary of required set of equations for any arbitrary transformation $x(r)$, 
which could be obtained similar to the LCR derivations \cite{KB10p022708}.
Let $P(r)$ be a wave function in $r$, then 
after the transformation $x(r)$, the corresponding equivalent $F(x)$ is given by
\beq
P(r) = F(x)/\sqrt{w(r)}, \ \  \ w(r) \equiv dx(r)/dr.
\eeq
If the equally spaced grid points are created in the $x$-variable, ${\bf x} = (x_1, x_2, ..., x_M)$,
the corresponding $r$-grid points become ${\bf r} = (r_1, r_2, ..., r_M)$ and 
the following equations could be used to convert required numerical calculations from the ${\bf r}$ to ${\bf x}$ grids.  
One-dimensional integration of function $f(r)$ becomes
\beq
\int_{r_1}^{r_M} dr f(r) = \int_{x_1}^{x_M} dx \frac{f(r)}{w(r)},
\eeq
where $r(x)$ is now a function of $x$ on the right-hand side of this and following equations. 
Matrix elements involving wave functions $P_i(r)$ and $P_j(r)$ and some radial function $f(r)$ become
\beq 
\int_{r_1}^{r_M} dr P_i P_j f(r) = \int_{x_1}^{x_M} dx \ F_i F_j \frac{f(r)}{[w(r)]^2},
\eeq
where, for example, $f(r)=1$ and $f(r)=V(r)$ yield the normalization and potential energy integrals, respectively.
Kinetic energy matrix elements require the second derivative of wave function $P(r)$, denoted by $P_{rr}\equiv d^2P(r)/dr^2$,
\beq
P_{rr}(r) = [w(r)]^{3/2} [F_{xx}(x)  + W(r) F(x)],
\eeq
where $F_{xx}\equiv d^2F(x)/dx^2$ and term with $F_x\equiv d F(x)/dx$ is absent. After defining 
$w_r\equiv d w(r)/dr$ and $w_{rr}\equiv d w_r/dr$, W(r) is given by
\beq
W(r) = -\frac{1}{2} \left[w_{rr} -\frac{3(w_r)^2}{2w}\right] w^{-3},
\eeq
where in the case of MCR grid 
\beq
w(r) = a + \frac{b}{y}, \ \ w_r = -\frac{b}{y^2}, \ \ w_{rr} = \frac{2b}{y^3}, 
\eeq
\beq
W_{\mbox{\tiny{MCR}}}(r) = -\frac{1}{2} \left[w_{rr} -\frac{3(w_r)^2}{2w}\right] w^{-3}.
\eeq


The kinetic energy matrix element becomes
\beq 
-\frac{1}{2}\int_{r_1}^{r_M} dr P_i P_{rr} =  -\frac{1}{2} \int_{x_1}^{x_M} dx F_i [F_{xx} +W F] ,
\eeq





The two-electron Slater integrals 



The corresponding LCR wave functions and radial integration rules are found by starting from the kinetic energy radial integral
\beq
K_{ij}=-\frac{1}{2} \int_0^{\infty} \dd r \
P_i(r) \left( \frac{\mbox{d}^2}{\mbox{d}r^2} - \frac{l(l+1)}{r^2} \right) P_j(r), \label{K_ij}
\eeq
and defining the LCR transformation function $f(x)$
which converts the radial wave functions $P_i(r)$ to the corresponding LCR functions $F_i(x)$ via
\beq
P_i \left( r(x) \right) = f(x) F_i(x), \label{P_i}
\eeq
where $r$ is now a function of $x$.


After substituting (\ref{P_i}) into (\ref{K_ij}), for the radial equations in the new variable $x$ to resemble one-dimensional motion (as for the $r$ or LR's $y$ coordinates), the term containing $\mbox{d}F(x)/\mbox{d}x$ must be eliminated arriving at
\beq
f(x) = z^{-1/2}, \ \ z=a+b/y, \ \ y=c+r(x), \label{f_x}
\eeq
\beq
\begin{array}{ll}
K_{ij}=-\frac{1}{2} \int_0^{\infty} & \mbox{d}x \ F_i(x)  \left( \frac{\mbox{d}^2}{\mbox{d}x^2} \right. \\
  & - \left. \left[ \frac{1}{4} + l(l+1) \frac{(c+r)^2}{r^2} \right] \right) F_j(x).
\end{array}
\label{K_ij_LCR}
\eeq
Other one-electron matrix elements of any operator $Q(r)$, e.g. $Q(r)=V(r)$ or $Q(r)=1$,
\beq
Q_{ij}= \int_0^{\infty} \dd r \ P_i(r) P_j(r) Q(r), \label{Q_ij}
\eeq
become
\beq
Q_{ij}= \int_0^{\infty} dx \  z^{-2} F_i(x) F_j(x) Q(r). \label{Q_ij_LCR}
\eeq


The LCR transformation is also consistent with the Hartree algorithm \cite{H57} of evaluating the so-called two-electron Slater integrals,
\beq
\begin{array}{ll}
R^k(a b, a' b')= & \int_0^{\infty} \dr_1 \int_0^{\infty} \dr_2 P_a(r_1) P_b(r_2) \\
 & \times \frac{r^k_{<}}{r^{k+1}_{>}} P_{a'}(r_1) P_{b'}(r_2),
\end{array}
\label{R_k}
\eeq
using differential equations, where $r_{<}=\min(r_1, r_2)$ and $r_{>}=\max(r_1, r_2)$. In the $r$-space, the Hartree algorithm re-defines $R^k(a b, a' b')$ as
\beq
R^k(a b, a' b')= \int_0^{\infty} P_a(r) P_{a'}(r)
 \frac{1}{r} Y^k_{b b'}(r) \ \dr,
\label{R_k_from_Y}
\eeq
\beq
Y^k_{ab}(r) = Z^k_{ab}(r)
+ \int_r^{\infty} \left( \frac{r}{s} \right)^{k+1} P_a(s) P_{b}(s) \ \dd s,
\label{Y_k}
\eeq
\beq
Z^k_{ab}(r) = \int_0^r \left( \frac{s}{r} \right)^k P_a(s) P_{b}(s) \ \dd s
\label{Z_k}
\eeq
and numerically solves a pair of differential equations for $Y^k$ and $Z^k$,
\beq
\frac{\dd}{\dd r} Z^k_{ab}(r) = P_a(r) P_{b}(r) - \frac{k}{r} Z^k_{ab}(r), \label{Z_k_eq} \eeq
\beq
\frac{\dd}{\dd r} Y^k_{ab}(r) = \frac{1}{r} \left[ (k+1) Y^k_{ab}(r) - (2k+1) Z^k_{ab}(r) \right],
\label{Y_k_eq}
\eeq
with the boundary conditions $Z^k_{ab}(0)=0$ and $Y^k_{ab}(r) \rightarrow Z^k_{ab}(r)$ as $r \rightarrow \infty$.


After the LR transformation the Hartree algorithm takes the form \cite{FFBJ97} of
\beq
\frac{\dd}{\dd y} \left( r^k Z^k_{ab}(y) \right) = r^{k+2} \bar{P}_a(y) \bar{P}_{b}(y), \label{Z_k_eq_LR} \eeq
\beq
\frac{\dd}{\dd y} \left( r^{-(k+1)} Y^k_{ab}(y) \right) = -(2k+1)r^{-(k+1)} Z^k_{ab}(y),
\label{Y_k_eq_LR}
\eeq
where $\bar{P}_a(y)=P_a(r)/\sqrt{r}$.


In the case of the LCR transformation, the same equations become
\beq
\frac{\dd}{\dd x} \left( r^k Z^k_{ab}(x) \right) = (c+r)^2r^k F_a(x) F_{b}(x), \label{Z_k_eq_LCR} \eeq
\beq
\frac{\dd}{\dd x} \left( r^{-(k+1)} Y^k_{ab}(x) \right) = -(2k+1)(c+r)r^{-(k+2)} Z^k_{ab}(x),
\label{Y_k_eq_LCR}
\eeq
where the existing computer code for $Y^k$ \cite{FF87} needs to be only slightly modified without affecting the accuracy nor stability of the numerical procedure.

And finally, the Numerov algorithm, e.g. \cite{JS02}, could still be used  via
\beq
\left[ \frac{\dd^2}{\dd x^2} - \frac{1}{4} + 2(E-V(r)) (c+r)^2 \right] F(x) = 0
\eeq
and was used to calculate the Coulomb continuum wave functions in this work.





\section{RESULTS}

\subsection{Resonances in $e$-He $S$-wave scattering}

see the resulting energy levels for two $\lambda_{\rm L}$ optimized with $n_\gamma=5$ and $n_\gamma=7$.
Note that $e^{\mbox{\tiny{DHIF}}}(1s2s,^1S)=-2.14418810$ from \cite{DHIF94} is inconsistent
and it is likely an error $-2.14419810$?


TODO Error in exact energy triplet?

[TODO]
Atomic unit of energy (or Hartree) was set to 27.21138386 eV \cite{MTN08}. A tabular
form of the JM and CCC cross sections is available from jmatrix.googlecode.com.


Cross sections are used to verify the negative-ion eigenstates responsible for resonances.
Verification is done by excluding a particular negative-ion state from the JM calculation and observing disappearance of the resonance behavior.
Once identified, their radial electron distributions are used to classify the resonances relative to the target eigenstates.






Again, both CCC and JM methods described the target helium atom
, where the target eigenstates were constructed from the first $N_t$ JM functions (\ref{psi_H_psi}). Convergence in the CCC cross sections
(Figs.~\ref{Fig_He_n2} and \ref{Fig_He_n3}) was achieved at $N_t=?$, where the corresponding JM cross sections
converged at $N_t=?$ and $N=?$.




\begin{table}[htb]
\caption{\label{Tab_ENGS}
Energy levels (a.u.) and excitation thresholds (eV) of the first nine bound states of helium in the
$S$-wave model.
}
\begin{ruledtabular}
%\begin{tabular}{lcr}
\begin{tabular}{rlll}
Classification & threshold (eV) & Eigenvalues (a.u) & ($N_c$,$N_t$)   \\
\hline
$\mbox{He}(1s^2,^1S)$ & 0  & -2.879 028 569 1 &  (50,50)   \\ %
            error =   & 0.001 80 & -2.878 962 303   &  (7,30)    \\
            error =   & 0.015 94 & -2.878 442 699   &  (3,30)    \\
            error =   & 0.177 47 & -2.872 506 673   &  (1,30)    \\
\hline
$\mbox{He}(1s2s,^3S)$   & 19.178  & -2.174 264 856 3 & (50,50) \\  %-2.174 264 856 288154 -2.174264856287701, -2.0684901366080752,
                 &  & -2.174 264 618   &  (7,30)   \\
                &  & -2.174 245 504   &  (1,30)    \\
\hline
$\mbox{He}(1s2s,^1S)$     &  19.996 & -2.144 197 258 7 &  (50,50) \\ %-2.144 197 258 7 31818,
                          &  & -2.144 191 393   &  (7,30)   \\
                          &   & -2.143 449 321   &  (1,30)    \\
\hline
$\mbox{He}(1s3s,^3S)$     & 22.056  & -2.068 490 070   &  (7,30)   \\
                          &        & -2.068 484 660   &  (1,30)    \\
\hline
$\mbox{He}(1s3s,^1S)$     & 22.266   & -2.060 792 356   & (7,30)    \\
                          &          & -2.060 573 161   &  (1,30)    \\
\hline
$\mbox{He}(1s4s,^3S)$    & 22.928  & -2.036 438 560   &  (7,30)    \\
                         &         & -2.036 436 372   &  (1,30)   \\
\hline
$\mbox{He}(1s4s,^1S)$   &  23.011 & -2.033 392 203 &  (7,30)   \\
                        &         & -2.033 300 706 &  (1,30)   \\
\hline
$\mbox{He}(1s5s,^3S)$   &   23.305  & -2.022 583 695   &  (7,30)    \\
                        &           & -2.022 582 608   &  (1,30)    \\
\hline
$\mbox{He}(1s5s,^1S)$   &  23.346   & -2.021 079 423   &  (7,30)    \\
                        &              & -2.021 033 007   &  (1,30)    \\
\hline
$\mbox{He}^+(1s)$       &  23.920 & -2 	 &    Ionization \\

%nc50 -2.8790285691204813, -2.144197258731818, -2.060794037546691, -2.0333785522995016, -2.019784513192403,
%-2.174264856288154, -2.0684901366081845, -2.0364341783161515, -2.021798838036517,

%nc10   -2.878976298682523, -2.1441926558786752, -2.0607927208274255, -2.0333923539473497,
% -2.1742647128049755, -2.0684900966824507, -2.036438571423033,

%nc7 -2.8789623026211353, -2.1441913928180427, -2.0607923564616435, -2.0333922030334937, -2.0210794228323463,
% -2.1742646178307643, -2.068490069685457, -2.036438559573639, -2.022583694613397,

%nc1 -2.872506672905663, -2.143449321021511, -2.06057316103011, -2.033300705504141, -2.0210330065059448,
% -2.1742455043163393, -2.0684846598208337, -2.0364363721136964, -2.022582608219962,

\end{tabular}
\end{ruledtabular}
\end{table}


\section{CONCLUSIONS}


In order to achieve the stated goal we apply the $J$-matrix (JM) approach to electron-atom scattering,
which has been recently revised by merging it with the Fano's multi-configuration interaction matrix elements \cite{Fano65}.
In that preceding JM paper \cite{BF11},
the $S$-wave $e$-He scattering problem was solved within the frozen-core (FC) model of helium for
the elastic, $2^{1,3}S$-excitation, and single ionization cross sections for impact energies in the range 0.1-1000eV.
The reported in \cite{BF11} "proof-of-principle" JM calculations were in complete agreement with the convergent-close-coupling (CCC) method,
within the FC model.
In this sequel, the scattering target helium atom is described at much higher level of accuracy overcoming the FC model.
It is found that the theory in \cite{BF11} is sufficient to fully solve the $S$-model below the single ionization threshold.
The presented JM results (1-30 eV) are confirmed by the corresponding CCC calculations providing
total elastic, $2^{1,3}S$ and $3^{1,3}S$ excitation cross sections with a "benchmark"-level of accuracy for the first time for the considered $S$-wave model.


\begin{acknowledgments}
This work was supported by the Australian Research Council. IB
acknowledges the Australian National Computational Infrastructure
Facility and its Western Australian node iVEC.
\end{acknowledgments}



\bibliographystyle{apsrev}
\bibliography{../bibtex/qm_references}

\end{document}
