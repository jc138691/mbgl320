%%\documentclass[aip,jcp]{revtex4-1}
\documentclass[aip
, pra
, showpacs
, aps
, twocolumn
%, onecolumn
, groupedaddress
, floatfix
%, preprint
]{revtex4}
%]{revtex4-1}
\usepackage{graphicx, amsbsy, bm, dcolumn, amsmath}

\usepackage{currfile}
\usepackage{datetime}

\newcommand{\etal}{{\em et~al.\/}}
\newcommand{\beq}{\begin{equation}}
\newcommand{\eeq}{\end{equation}}
\newcommand{\barr}{\begin{array}}
\newcommand{\earr}{\end{array}}
\newcommand{\bfr}{{\bf r}}
\newcommand{\bfk}{{\bf k}}
\newcommand{\dd}{\mbox{d}}
\newcommand{\dr}{\mbox{d} r}
\newcommand{\vare}{\varepsilon}
\newcommand{\calN}{{\cal N}}
\newcommand{\di}{_{\mbox{\tiny{di}}}}
\newcommand{\ex}{_{\mbox{\tiny{ex}}}}



\newcommand{\isum}%
{\mathop{\hbox{$\displaystyle\sum\kern-13.2pt\int\kern1.5pt$}}}

\begin{document}

\title {The ionization SDCS coupled-channels paradox or reality?}

\author{D. A. Konovalov}
\affiliation{Discipline of Information Technology, School of Business, James Cook University, Townsville, Queensland 4811, Australia}

\author{I. Bray}
\affiliation{ARC Centre for Antimatter-Matter Studies,
Curtin University, GPO Box U1987, Perth, Western Australia 6845, Australia}



\date{\today}
\date{\today, \currenttime}

\begin{abstract}
todo
\end{abstract}

\pacs{34.80.Dp} %34.80.Dp	Atomic excitation and ionization
\maketitle
This file is: \currfilename.



\section{INTRODUCTION}
Initial state $\Psi_i$, $q=(\bfr, \nu)$, $a$-atom, $b$-beam.
Before the collision, the target and scatting electrons are in the spin states $| \chi_{\mu_a} \rangle$ and $| \chi_{\mu_b} \rangle$, respectively.
After the collision, each of the two electron are detected with $| \chi_{\mu_\alpha} \rangle$ and momentum $\bfk_\alpha$, $\alpha=\{1,2\}$.
\beq
\Psi_i(q_1, q_2) = \chi_{\mu_a}(\nu_1) \chi_{\mu_b}(\nu_1) \phi_0(\bfr_1) \psi_{\bfk_0}(\bfr_2) 
\eeq
\beq
\Psi_f(q_1, q_2) = \chi_{\mu_1}(\nu_1) \chi_{\mu_2}(\nu_1) \psi_{\bfk_1}(\bfr_1) \psi_{\bfk_2}(\bfr_2)
\eeq
\beq
\chi_{\mu}(\nu) =\delta_{\mu \nu}
\eeq

The SDCS ionization amplitude is given
\beq
f_{fi} = \delta_{\mu_a \mu_1} \delta_{\mu_b \mu_2} V(\bfk_1,\bfk_2; \bfk_0),
\eeq
\beq
V(\bfk_1,\bfk_2; \bfk_0) = \int \psi_{\bfk_1}(\bfr_1) \psi_{\bfk_2}(\bfr_2) V(\bfr_1, \bfr_2) \phi_0(\bfr_1) \psi_{\bfk_0}(\bfr_2),
\eeq

CC-version
\beq
\Psi_S(q_1, q_2) = \langle \nu_1 \nu_2 | S \rangle \phi_0(\bfr_1) \psi_{\bfk_0}(\bfr_2)
\eeq

\section{THEORY}




\section{CONCLUSIONS}



\begin{acknowledgments}
This work was supported by the Australian Research Council. IB
acknowledges the Australian National Computational Infrastructure
Facility and its Western Australian node iVEC.
\end{acknowledgments}



\bibliographystyle{apsrev}
\bibliography{../bibtex/qm_references}



\end{document}
