%%\documentclass[aip,jcp]{revtex4-1}
\documentclass[aip
, pra
, showpacs
, aps
, twocolumn
%, onecolumn
, groupedaddress
, floatfix
%, preprint
]{revtex4}
%]{revtex4-1}
\usepackage{graphicx, amsbsy, bm, dcolumn, amsmath}

\newcommand{\etal}{{\em et~al.\/}}
\newcommand{\beq}{\begin{equation}}
\newcommand{\eeq}{\end{equation}}
\newcommand{\barr}{\begin{array}}
\newcommand{\earr}{\end{array}}
\newcommand{\vecr}{{\bf r}}
\newcommand{\JM}{\mbox{\tiny{JM}}}
\newcommand{\KB}{\mbox{\tiny{KB}}}
\newcommand{\vare}{\varepsilon}


\newcommand{\isum}%
{\mathop{\hbox{$\displaystyle\sum\kern-13.2pt\int\kern1.5pt$}}}

\begin{document}

\title {Electron-atom scattering theory via a short-range $L^2$ basis}

\author{Dmitry A. Konovalov}
\affiliation{Discipline of Information Technology, School of Business, James Cook University, Townsville, Queensland 4811, Australia}

\author{Igor Bray}
\affiliation{ARC Centre for Antimatter-Matter Studies,
Curtin University, GPO Box U1987, Perth, Western Australia 6845, Australia}


\date{\today}

\begin{abstract}
TODO

\end{abstract}

\pacs{34.80.Dp} %34.80.Dp	Atomic excitation and ionization
\maketitle

\section{INTRODUCTION}
TODO

\section{THEORY}
\subsection{Short-range potential scattering}
After the partial-wave expansion and retaining only the $s$-wave terms $(l=0)$, the potential scattering problem is reduced \cite{LL85, Taylor72} to solving the radial Schr\"odinger equation
\beq
(H-E) | \Psi_E \rangle =0, \ \ \ E=k^2/2,
\eeq
\beq
\Psi_E(r \rightarrow \infty) \sim \sin(kr+\delta_0), \ \ \ \Psi_E(r \rightarrow 0)=0,
\eeq
\beq
H=K+V(r), \ \ \ K=-\frac{1}{2} \frac{\mbox{d}^2}{\mbox{d}r^2}, \label{K}
\eeq
where $\delta_0$ is the $s$-wave phase shift and $E$ is the total energy of the system.

\beq
\langle \xi_m | \xi_{m'} \rangle=\delta_{mm'}, \ \ \ m,m'=0,1,...,\infty,
\eeq
\beq
\langle \Psi_i |H| \Psi_j \rangle = E_i \delta_{ij} , \ \ \ \langle \Psi_i | \Psi_j \rangle=\delta_{ij}. 
\eeq
\beq \barr{l}
\Psi_i^{\JM}(r) = \sum_{m=0}^{N-1} A_{im} \phi_m(r), \\
\Psi_i^{\KB}(r) = \sum_{m=0}^{N-1} B_{im} \xi_m(r),
\earr \eeq
\beq
\hat{P}_{\KB} = \sum_{m=0}^{N-1} | \xi_m \rangle \langle \xi_m |.
\eeq


To help explain the logic, the new method (denoted KB) is presented together with the original JM method,
\beq 
\Psi_E^{\JM}(r) =\sum_j \Psi_j^{\JM}(r)a_j  + \sum_{p=N}^{\infty} \phi_p(r) f_p, 
\eeq
\beq 
f_p = s_p + R_0 c_p, \ \ \ R_0 = \tan \delta_0,
\eeq
\beq \barr{l}
\Psi_E^{\KB}(r) =\sum_j \Psi_j^{\KB}(r) b_j  + c_0 \psi_1(r)  + s_0 \psi_2(r) ,\\
c_0 = \cos(\delta_0), \ \ s_0 = \sin(\delta_0),\\
\psi_1(r) = (1-\hat{P}_{\KB}) \sin(kr),\\
\psi_2(r) = (1-\hat{P}_{\KB}) \cos(kr),
\earr \eeq
\beq
\psi_1(r) c_0 + \psi_2(r) s_0 =
(1-\hat{P}_{\KB}) \sin(kr + \delta_0),\\
\eeq
where $\Delta_0 = \exp(i\delta_0)$, 
$a_m$ are the {\em inner}-space expansion coefficients describing  the electron's interaction with  $V(r)$, and
$R_0$ is the $s$-wave term of the reactance matrix (also known as the $K$ matrix \cite{Taylor72}).


The unknown $a_m$ and $R$ are found by simultaneously solving
\beq \barr{rl}
JM: & \langle\Psi_i^{\JM}|H-E|\Psi_E^{\JM}\rangle= 0, \\
KB: & \langle\Psi_i^{\KB}|H-E|\Psi_E^{\KB}\rangle=0 ,
\earr \eeq

\beq \barr{rl}
JM: & a_i = - \sum_{p=N}^{\infty} X_{ip} f_p / (E_i-E),\\
& X_{ip}=  \langle\Psi_i^{\JM}|H-E| \phi_p \rangle ,\\
& G_{pp'}= \sum_j X_{jp} X_{jp'}/(E_j-E),
\earr \eeq

\beq \barr{rl}
KB: & b_i = - (S_i c_0 + C_i s_0) / (E_i-E), \\
& S_i = \langle\Psi_i^{\KB}|H-E|  \psi_1 \rangle ,\\
& C_i = \langle\Psi_i^{\KB}|H-E|  \psi_2 \rangle ,\\
& g_1= \sum_j S_{j} S_{j}/(E_j-E),\\
& g_2= \sum_j S_{j} C_{j}/(E_j-E),
\earr \eeq

\beq \barr{rl}
JM: & \langle\phi_p|H-E|\Psi_E^{\JM}\rangle=0 , \ \ \ p \geq N.\\
KB: & \langle \psi_1 |H-E|\Psi_E^{\JM}\rangle=0,
\earr \eeq

\beq \barr{rl}
JM: & \sum_j X_{jp} a_j = - \sum_{p'=N}^{\infty} W_{pp'} f_p',\\
& W_{pp'}  = \langle \phi_p | H-E | \phi_{p'} \rangle,\\
& \sum_{p'=N}^{\infty} f_{p'}  \sum_j G_{jp,jp'} = \sum_{p'=N}^{\infty} W_{pp'} f_{p'},\\
\earr \eeq

\beq \barr{rl}
KB: & \sum_j S_{j} b_j = - (w_1 c_0 +  w_2 s_0), \\
 & w_1  = \langle \psi_1 |H-E| \psi_1 \rangle,\\
 & w_2 = \langle \psi_1 |H-E| \psi_2 \rangle,\\
 & g_1 c_0 + g_2 s_0 = w_1 c_0 +  w_2 s_0,\\
 & R = -(w_1 - g_1) / (w_2 - g_2),
\earr \eeq

where the $s$-wave partial $S$-matrix and elastic cross section are given by
\beq
S_{00}=(1+\mbox{i}R)(1-\mbox{i}R)^{-1},
\eeq
\beq
\sigma_{00}=\frac{\pi}{k^2} |S_{00}-1|^2,
\eeq
respectively, and where
\beq
(WSJS) = W s_N J_{N,N-1} + s_{N-1},
\eeq
\beq
(WCJC) = W c_N J_{N,N-1} + c_{N-1},
\eeq
\beq
W= \sum_{m=0}^{N-1} \frac{D_{m,N-1}^2}{e_m-E}.
\label{g}
\eeq

\subsection{Coulomb-tail potential scattering}



\section{CONCLUSIONS}
TODO


\begin{acknowledgments}
\end{acknowledgments}



\bibliographystyle{apsrev}
\bibliography{..//bibtex//qm_references}
%\bibliography{qm_references}
\end{document}
