%%\documentclass[aip,jcp]{revtex4-1}
\documentclass[aip
, pra
, showpacs
, aps
, twocolumn
%, onecolumn
, groupedaddress
, floatfix
%, preprint
]{revtex4}
%]{revtex4-1}
\usepackage{graphicx, amsbsy, bm, dcolumn, amsmath}

\newcommand{\etal}{{\em et~al.\/}}
\newcommand{\beq}{\begin{equation}}
\newcommand{\eeq}{\end{equation}}
\newcommand{\barr}{\begin{array}}
\newcommand{\earr}{\end{array}}
\newcommand{\vecr}{{\bf r}}
\newcommand{\JM}{\mbox{\tiny{JM}}}
\newcommand{\KB}{\mbox{\tiny{KB}}}
\newcommand{\vare}{\varepsilon}


\newcommand{\isum}%
{\mathop{\hbox{$\displaystyle\sum\kern-13.2pt\int\kern1.5pt$}}}

\begin{document}

\title {Calculation of excitation and single-ionization amplitudes using a short-range $L^2$ basis}

\author{Dmitry A. Konovalov}
\affiliation{ARC Centre for Antimatter-Matter Studies}
\affiliation{Discipline of Information Technology, School of Business,
James Cook University, Townsville, Queensland 4811, Australia}

\author{Igor Bray}
\affiliation{ARC Centre for Antimatter-Matter Studies,
Curtin University, GPO Box U1987, Perth, Western Australia 6845, Australia}


\date{\today}

\begin{abstract}
It is shown that existing JM method could be significantly simplified without the loss of convergence rate or accuracy.

\end{abstract}

\pacs{34.80.Dp} %34.80.Dp	Atomic excitation and ionization
\maketitle

\section{INTRODUCTION}
TODO

\section{THEORY}
\subsection{One-channel scattering}


After the partial-wave expansion \cite{Taylor72,N82} and focusing only on the $s$-wave terms $(l=0)$,
electron scattering on a spherically symmetric potential $V(r)$
could be described by the time-independent radial Schr\"odinger equation
\beq
(H-E) | \Psi_E \rangle =0,  \ \ \ H = H_0 + V, \label{H_E_Psi_E}
\eeq
where $E=k^2/2$ and $k$ are the incident electron energy and momentum, respectively.
Let $\psi_k$ and $\widehat{\psi}_k$ be real regular and irregular solutions for $H_0$,
\beq \barr{l}
(H_0-E) | \psi_k \rangle =0,  \ \ \ \psi_k(r=0) = 0,\\
(H_0-E) | \widehat{\psi}_k \rangle =0,  \ \ \ \widehat{\psi}_k(r=0) \neq 0,\\
\earr \label{H_0_E_psi}\eeq
then for the short-range potential $V(r)$,
\beq \barr{l}
H_0 = \hat{K},\ \ \ \hat{K} = -(1/2) d^2/dr^2,\\
\psi_k(r) = \sin(kr) \sqrt{2/k},\\
\widehat{\psi}_k(r) = \cos(kr) \sqrt{2/k},
\earr \label{K} \eeq
where the normalization $\sqrt{2/k}$ will become convenient in the multichannel scattering.
If present, the long-range Coulomb potential $V_c=-Z/r$ is included into $H_0$
and $\psi_1$ and $\psi_2$ become the corresponding Coulomb wave functions satisfying Eq.~(\ref{H_0_E_psi})
with $H_0 = \hat{K} + V_c$.



In 1967, \citet{Harris67} showed that Eq.~(\ref{H_E_Psi_E}) could be solved
approximately using a square-integrable ($L^2$) basis.
Let $\{\xi_p(r)\}_{p=0}^\infty$ be an $L^2$-complete set of real orthonormal functions
such that for any $p,p'=0,1,...,\infty$
\beq
\langle \xi_p | \xi_{p'} \rangle=\delta_{pp'}, \ \ \ \xi_p(r=0)=0,
\eeq
of which the first $N$ basis functions are used to diagonalize $H$ arriving at
\beq
\langle \Psi_i |H| \Psi_j \rangle = E_i \delta_{ij} , \ \ \langle \Psi_i | \Psi_j \rangle=\delta_{ij},
\eeq
\beq
\Psi_i(r) = \sum_{p=0}^{N-1} C_{ip} \xi_p(r).
\eeq
Then $\Psi_E(r)$ is approximated at the system's eigenvalues $\{E_i\}_{i=1}^{N}$ as
\cite{Harris67}
\beq \barr{l}
\Psi_{E_i}(r) = a_i \Psi_i(r)  + [\phi_{k_i}(r)  + K_0 \widehat{\phi}_{k_i}(r)],\\
K_0 = \tan{\delta_0},
\earr \label{Psi_EES} \eeq
where the {\em standing}-wave (equal flax of {\em incoming}- and {\em outgoing}-waves) boundary conditions are satisfied
\beq
\Psi_E(r \rightarrow \infty) \sim  \sin(kr+\delta_0) \sqrt{2/k}, \ \ \ \Psi_E(r= 0)=0,
\eeq
and where $\phi_{\{1,2\}}$ are related to $\psi_{\{1,2\}}$ and will be defined later.
The $s$-wave elastic scattering results could be expressed via:
phase shift ($\delta_0$), $K$-matrix ($K_0$), $S$-matrix ($S_{0}$) and cross section ($\sigma_{0}$)
as follows
\beq \barr{l}
S_{0}=(1+\mbox{i}K_0)(1-\mbox{i}K_0)^{-1}, \\
\sigma_{0}=\pi k^{-2} |S_{0}-1|^2.
\earr \eeq


Eq.~(\ref{Psi_EES}) is solved by
\beq \barr{l}
\langle\Psi_i|H-E_i|\Psi_{E_i}\rangle=0,\\
K_{00} = - s / c, \\
s = \langle\Psi_i|H-E_i|  \phi_{k_i} \rangle \equiv \langle\Psi_i|H|  \phi_{k_i} \rangle, \\
c = \langle\Psi_i|H-E_i|  \widehat{\phi}_{k_i} \rangle \equiv \langle\Psi_i|H|  \widehat{\phi}_{k_i} \rangle.
\earr \eeq
Once $K_0$ is known, $a_i$ is found from
\beq \barr{l}
\langle \xi_{N-1} | \Psi_{E_i}\rangle
= a_i \langle \xi_{N-1}| \Psi_i \rangle = ,\\
\ \ \ =  \langle \xi_{N-1}| \psi_{k_i} \rangle
+ K_0 \langle \xi_{N-1}| \widehat{\psi}_{k_i} \rangle ,\\
\earr \eeq
\beq
K_0 = -\langle \psi_{k_i} | V | \Psi_{E_i} \rangle.
\eeq


The goal here is to improve upon the Harris' {\em Expansion Approach} \cite{Harris67} to the extent that the Eq.~(\ref{Psi_EES})
converges to the exact solution as $N$ increases.
To achieve the stated goal, this Eigenstate Expansion for Scattering (EES) method relies on the following conjecture.
Within the functional space of $N$ basis $L^2$ functions  $\{\xi_p(r)\}_{p=0}^N$,
the eigenstates $\{\Psi_i\}_{i=1}^{N}$ are the best possible solutions of the scattering problem in Eq.~(\ref{H_E_Psi_E})
with the standing-wave boundary conditions at the corresponding eigenvalues $\{E_i\}_{i=1}^{N}$.



Accepting the conjecture, it immediately follows that the {\em inner} functional space must be removed from
\beq \barr{l}
\phi_k(r) = (1-\hat{P}_N) \psi(r),\\
\widehat{\phi}_(r) = (1-\hat{P}_N) [\widehat{\psi}(r) - \exp{(-\lambda r)}],
\earr \eeq
\beq
\hat{P}_N = \sum_{p=0}^{N-1} | \xi_p \rangle \langle \xi_p |.
\eeq



\subsection{Multichannel scattering}
\cite{CA73} - only elastic
\cite{NO72},
\cite{TF79},
\cite{Nesbet78}, \cite{Lucchese86}

\beq
\langle \Phi_\gamma|H_t|\Phi_{\gamma'}\rangle=e_\gamma \delta_{\gamma\gamma'}, \ \
 \langle \Phi_\gamma|\Phi_{\gamma'}\rangle=\delta_{\gamma\gamma'},
 \label{psi_H_psi} \eeq
\beq
\langle\Psi_i^\Gamma|H|\Psi_j^\Gamma\rangle=E_i\delta_{ij}, \ \
 \langle\Psi_i^\Gamma|\Psi_j^\Gamma\rangle=\delta_{ij},
\label{Psi_H_Psi} \eeq


\beq \barr{l}
 \Psi_{E_i}^{\Gamma}  =  a \Psi_i^{\Gamma}
 + A^{\Gamma} \Phi_{\gamma}  \phi_{\gamma}
+ \sum_{\gamma''} K_{\gamma''} A^{\Gamma} \Phi_{\gamma''}   \widehat{\phi}_{\gamma''} ,
\earr \label{Psi_} \eeq
\beq
\phi_{\gamma}\equiv \phi_{k_\gamma}, \ \ \ \psi_{\gamma}\equiv \psi_{k_\gamma}, \ \ \   k_{\gamma}^2 = 2(E_i - e_\gamma),
\eeq


\beq \barr{l}
\langle\Psi_i^\Gamma|H-E_i|\Psi_{E_i}^{\Gamma}\rangle=0,\\
\sum_{\gamma''}  C_{\gamma''} K_{\gamma''}  = -s
, \ \ \ ({\bf C}^t {\bf K}) = -s,\\
s = \langle \Psi_i^\Gamma |HA^{\Gamma} |\Phi_{\gamma} \phi_{\gamma} \rangle,\ \
C_{\gamma'} =  \langle \Psi_i^\Gamma |HA^{\Gamma} |\Phi_{\gamma'} \widehat{\phi}_{\gamma'} \rangle,
\earr \label{Psi_} \eeq

\beq \barr{l}
K_{\gamma'} = - \langle \Phi_{\gamma'}  \psi_{\gamma'}
| V | \Psi_{E_i}^{ \Gamma} \rangle,\\
K_{\gamma'} =   - [a Y_{\gamma'}   + B_{\gamma'}
+ \sum_{\gamma''}  W_{\gamma' \gamma''} K_{\gamma''}],\\
{\bf K} =   -[a {\bf Y} + {\bf B} + {\bf W K}],\\
{\bf K}  = -({\bf I}+{\bf W})^{-1}   (a {\bf Y}    + {\bf B}),
\earr \eeq

\beq \barr{l}
s = a y  + b,   \\
a = (s - b) / y,\\
y = {\bf C}^t ({\bf I}+{\bf W})^{-1}  {\bf Y}, \\
b = {\bf C}^t ({\bf I}+{\bf W})^{-1}  {\bf B},
\earr \eeq
\beq \barr{l}
Y_{\gamma'} = \langle \Phi_{\gamma'}  \psi_{\gamma'} |V |\Psi_i^{\Gamma} \rangle,\\
B_{\gamma'} = \langle \Phi_{\gamma'}  \psi_{\gamma'} |V A^{\Gamma} |\Phi_{\gamma} \phi_{\gamma} \rangle,\\
W_{\gamma' \gamma} = \langle \Phi_{\gamma'}  \psi_{\gamma'} |V A^{\Gamma} |\Phi_{\gamma} \widehat{\phi}_{\gamma} \rangle,\\
\earr \label{Psi_} \eeq

\subsection{$S$-wave $e$-H scattering}



\beq \barr{l}
Y_{\gamma'} = \langle \Phi_{\gamma'}  \psi_{\gamma'} |V|\Psi_i^{\Gamma} \rangle,\\
B_{\gamma'} = \langle  \Phi_{\gamma'}  \psi_{\gamma'} |VA^{\Gamma} |\Phi_{\gamma} \phi_{\gamma} \rangle,\\
W_{\gamma' \gamma} = \langle \Phi_{\gamma'}  \psi_{\gamma'} |V A^{\Gamma} |\Phi_{\gamma} \widehat{\phi}_{\gamma} \rangle,\\
\earr \label{Psi_} \eeq

\beq \barr{l}
\langle  \Phi(r_1) \psi(r_2) |f(r_1) g(r_2) - f(r_2) g(r_1) \rangle = \\
=\langle  \Phi |f \rangle  \langle  \psi |g \rangle -
\langle  \Phi |g \rangle  \langle  \psi |f \rangle
\earr \eeq


\section{CONCLUSIONS}
TODO


\begin{acknowledgments}
\end{acknowledgments}



\bibliographystyle{apsrev}
\bibliography{..//bibtex//qm_references}
%\bibliography{qm_references}
\end{document}
