%%\documentclass[aip,jcp]{revtex4-1}
\documentclass[aip
, pra
, showpacs
, aps
, twocolumn
%, onecolumn
, groupedaddress
, floatfix
%, preprint
]{revtex4}
%]{revtex4-1}
\usepackage{graphicx, amsbsy, bm, dcolumn, amsmath}

\newcommand{\etal}{{\em et~al.\/}}
\newcommand{\beq}{\begin{equation}}
\newcommand{\eeq}{\end{equation}}
\newcommand{\barr}{\begin{array}}
\newcommand{\earr}{\end{array}}
\newcommand{\vecr}{{\bf r}}
\newcommand{\dd}{\mbox{d}}
\newcommand{\dr}{\mbox{d} r}
\newcommand{\vare}{\varepsilon}
\newcommand{\calN}{{\cal N}}
\newcommand{\di}{_{\mbox{\tiny{di}}}}
\newcommand{\ex}{_{\mbox{\tiny{ex}}}}



\newcommand{\isum}%
{\mathop{\hbox{$\displaystyle\sum\kern-13.2pt\int\kern1.5pt$}}}

\begin{document}

\title {$J$-matrix method for Electron-Atom Ionization Amplitude}

\author{Dmitry A. Konovalov}
\affiliation{Discipline of Information Technology, School of Business, James Cook University, Townsville, Queensland 4811, Australia}

\author{Dmitry V. Fursa}
\affiliation{ARC Centre for Antimatter-Matter Studies,
Curtin University, GPO Box U1987, Perth, Western Australia 6845, Australia}

\author{Alisher S. Kadyrov}
\affiliation{ARC Centre for Antimatter-Matter Studies,
Curtin University, GPO Box U1987, Perth, Western Australia 6845, Australia}

\author{Igor Bray}
\affiliation{ARC Centre for Antimatter-Matter Studies,
Curtin University, GPO Box U1987, Perth, Western Australia 6845, Australia}

\date{\today}

\begin{abstract}


\end{abstract}

\pacs{34.80.Dp} %34.80.Dp	Atomic excitation and ionization
\maketitle


\section{INTRODUCTION}

\cite{KMSB03l}


\section{THEORY}

\subsection{Wave functions}
Let the nonorthogonal Laguerre functions used in the original JM method \cite{HY74p1201,BR76p1491}
be referred to as the JM functions and denoted by $\{\xi_p(r)\}_{p=0}^\infty$,
\[
\xi_p(r) = x^{l+1} \mbox{e}^{-x /2}
L_p^{2l+1}(x), \ \ \ p = 0, 1, ..., \infty,
\]
where $x=\lambda_{\rm L} r$,  $\lambda_{\rm L}$ is the Laguerre exponential falloff,
$l \equiv 0$ (for the $S$-model), and $L_p^{\alpha}(x)$ are the associated Laguerre polynomials \cite{abramowitz}.
The JM method splits the one-electron radial functional space into {\em inner} $\{\xi_p\}_{p=0}^{N-1}$
and {\em outer} $\{\xi_p\}_{p=N}^\infty$
subsets controlled by the number ($N$) of JM functions in the inner subset \cite{HY74p1201,BR76p1491}.


It is computationally convenient to use the corresponding {\em orthogonal} Laguerre functions,
hereafter referred to as the Laguerre functions (or basis) and denoted by $\{\hat{\xi}_p(r)\}_{p=0}^\infty$,
\[
\hat{\xi}_p(r) = C_p^{-1/2} x^{l+1} \mbox{e}^{-x /2}
L_p^{2l+2}(x), \ \ \ p = 0, 1, ..., \infty,
\]
\[
\int_0^{\infty} \mbox{d}r \ \hat{\xi}_p(r) \hat{\xi}_{p'}(r)=\delta_{pp'},
\]
where $C_p=(p+2l+2)!/(\lambda_{\rm L} p!)$.
Due to the following property of the Laguerre polynomials \cite{abramowitz},
\[
L_p^{\alpha + 1}(x)=\sum_{p'=0}^{p}L_{p'}^{\alpha}(x),
\]
for the purpose of the JM method, any $\xi_p$ could be replaced by the corresponding $\hat{\xi}_p$
within the JM's inner region of $p < N$ \cite{KB10p022708, KFB11}.
That is, the functional space created by $\{\xi_p\}_{p=0}^{N-1}$ is identical
to the one created by $\{\hat{\xi}_p\}_{p=0}^{N-1}$ and therefore $\{\xi_p\}_{p=0}^{N-1}$ are not required
in practice. Moreover, the scattering contribution of the outer subset is accounted for analytically within the JM method,
hence in fact none of the JM functions are calculated nor used \cite{KB10p022708, KFB11}.


The $N$  Laguerre functions from the inner functional space could now be used to construct $N$ {\em target orbitals}
$\{P_\lambda(r)\}_{\lambda=1}^{N}$.
The frozen-core model of helium was used in \cite{KFB11}, where it required the He$^+(1s)$ one-electron radial wave function
$P^+_{1s}(r)$ to be well approximated by one of the target orbitals, for example $P^+_{1s}(r) \approx P_1(r)$.
This was achieved in \cite{KFB11} by using first $N_t$ (mnemonic "t" is for target) Laguerre functions
\beq
P^+_\lambda(r)=\sum_{p=0}^{N_t-1} \hat{D}_{\lambda p} \hat{\xi}_p(r),
\ \ \ \langle P^+_\lambda|P^+_{\lambda'}\rangle=\delta_{\lambda\lambda'},
\label{P_ion_sum} \eeq
and diagonalizing the $S$-wave term of the one-electron Hamiltonian $h$,
\beq
h = -\frac{1}{2} \frac{d^2} {dr^2} - \frac{Z}{r},
\label{h}
\eeq
\beq
\langle P_\lambda^+|h|P^+_{\lambda'}\rangle=\varepsilon_\lambda \delta_{\lambda \lambda'},
\ \ \varepsilon_\lambda < \varepsilon_{\lambda+1},
\label{P_h_P}
\eeq
where atomic units are used throughout this manuscript,
$Z=2$, $N_t<N$, $1 \leq \lambda,\lambda' \leq N_t$,
and where superscript $'+'$ reminds that the resulting orbitals $\{P^+_\lambda\}_{\lambda=1}^{N_t}$ are eigenstates of helium one-electron ion and will be referred to as the {\em helium-ion} basis (or orbitals).
The remaining $\{P^+_\lambda\}_{\lambda=N_t+1}^{N}$ orbitals were directly assigned to the Laguerre functions in \cite{KFB11},
\beq
P^+_\lambda(r)= \hat{\xi}_{\lambda-1}(r),  \ \ \ N_t < \lambda \leq N,
\label{P_ion_N} \eeq
arriving at the first (considered here) target basis $\{P^+_\lambda\}_{\lambda=1}^{N}$.


Since the frozen-core model is not utilized in this study,
the one-electron diagonalization step [Eq.~(\ref{P_h_P})] is no longer essential
and the second (considered here) target basis is naturally given by just the Laguerre functions,
\beq
P_\lambda(r)= \hat{\xi}_{\lambda-1}(r),  \ \ \ 1 \leq \lambda \leq N.
\label{P_N} \eeq


The preceding part of this section describes how the two considered one-electron target bases are constructed.
Those are the helium-ion basis $\{P^+_\lambda\}_{\lambda=1}^{N}$ [Eqs.~(\ref{P_ion_sum}), (\ref{P_h_P}) and (\ref{P_ion_N})] and the Laguerre basis $\{P_\lambda\}_{\lambda=1}^{N}$ [Eq.~(\ref{P_N})]. Regardless of the chosen one-electron basis, the required two- and three-electron basis configurations are constructed as follows.
The $a$-electron square-integrable ($L^2$) wave functions
are obtained by diagonalizing the $S$-wave term of the total $a$-electron  Hamiltonian $H_a$,
\beq
H_{a} = \sum_{b=1}^{a} h_b + \sum_{b=1}^{a} \sum_{b'=1}^{b-1}
\frac{1}{\max{(r_{b}, r_{b'})}}. \\
\label{H_a}
\eeq
Specifically, the two-electron $\{ \psi_{\gamma} \}_{\gamma=1}^{N_2}$ and three-electron
$\{ \Psi^{\Gamma}_i \}_{i=1}^{N_3}$ wave functions  are given by
\beq
\langle\psi_\gamma|H_t|\psi_{\gamma'}\rangle=e_\gamma \delta_{\gamma\gamma'}, \ \
 e_\gamma < e_{\gamma+1},
\label{psi_H_psi}
\eeq
\beq
| \psi_\gamma \rangle= \sum_{\beta=1}^{N_2} C^t_{\gamma\beta} | \hat{A}\phi_\beta \rangle, \ \
\langle\psi_\gamma|\psi_{\gamma'}\rangle=\delta_{\gamma\gamma'},
\label{psi_H_psi_2}
\eeq
\beq
\langle\Psi_i^\Gamma|H|\Psi_{i'}^\Gamma\rangle=E_i\delta_{ii'},
\ \  E_i < E_{i+1},
\label{Psi_from_Fano} \eeq
\beq
| \Psi_i^\Gamma \rangle= \sum_{\beta=1}^{N_3} C_{i\beta} | \hat{A}\Phi_\beta^\Gamma \rangle,
 \ \ \langle\Psi_i|\Psi_{i'}\rangle=\delta_{ii'},
\label{Psi_from_Fano_2} \eeq
respectively,
where (i) $N_a$ is the size of the $a$-electron basis, $H_t \equiv H_{a=2}$, $H \equiv H_{a=3}$;
(ii) the eigenstates are labeled in ascending order of their eigenvalues; (iii) $\hat{A}$ is antisymmetrization operation;
(iv) $C^t_{\gamma\beta}$ and $C_{i \beta}$  are configuration-interaction (CI) coefficients;
(v) and where Fano's \cite{Fano65,KFB11} subshell-structured $a$-electron configurations $\{ \phi_{\beta} \}_{\beta=1}^{N_2}$
or $\{ \Phi_\beta^\Gamma \}_{\beta=1}^{N_3}$
are labeled in some arbitrary but fixed order.


One of the advantages of the JM method is its computational efficiency.
For example, so far in our last two JM papers \cite{KB10p022708, KFB11},
the JM results were obtained on a consumer-grade personal computer using freely available Java programming language \cite{JMatrixWebsite}.
In an attempt to maintain that computational "affordability" (and exclusively for that purpose),
the number of permitted "core" orbitals ($N_c$) of the first target electron is introduced as
an optimization parameter,
where $N_c=1$ corresponds to the frozen-core model in \cite{KFB11}.
Given the three interacting electrons of the considered problem, $N_c$, $N_t$ and $N$ control
how the two-electron $\phi_\beta$ and three-electron $\Phi_\beta^\Gamma$ basis configurations are constructed via
\beq
| \phi_\beta \rangle = | \lambda_{\beta 1} \lambda_{\beta 2} \rangle, \ \ \lambda_{\beta 1} \leq \lambda_{\beta 2},
\eeq
\beq
| \Phi_\beta^\Gamma \rangle = | \lambda_{\beta 1} \lambda_{\beta 2} \lambda_{\beta 3} \rangle,
\ \ \lambda_{\beta 1} \leq \lambda_{\beta 2} \leq \lambda_{\beta 3},
\eeq
\beq
1 \leq \lambda_{\beta 1} <= N_c, \ \ N_c \leq N_t,
\eeq
\beq
1 \leq \lambda_{\beta 2} <= N_t, \ \ N_t < N,
\eeq
\beq
1 \leq \lambda_{\beta 3} <= N,
\eeq
where each used $\lambda$ describes an electron occupying a subshell orbital
with $P_\lambda(r)$ (or $P^+_\lambda(r)$) as its radial component.
Note that the preceding parametrization implies (not shown here) the relevant spin coupling and subshell antisymmetrization
as per the general procedure for building $a$-electron subshell-based configurations from \cite{KFB11}.


Once $N_c$, $N_t$ and $N$ are selected,
all combinatorially possible unique configurations are included into the $\{ \phi_{\beta} \}_{\beta=1}^{N_2}$
and $\{ \Phi_\beta^\Gamma \}_{\beta=1}^{N_3}$ bases,
subject to spin coupling and antisymmetrization rules.
For example, the total number $N_2$, the numbers of singlet $N_2({^1S})$ and triplet $N_2({^3S})$ two-electron configurations become
\beq
N_2=N_2({^1S}) + N_2({^3S}),
\label{N_2}
\eeq
\beq
N_2({^1S})=N_c(N_c+1)/2+ N_c (N_t-N_c),
\label{N_1S}
\eeq
\beq
N_2({^3S})=N_c(N_c-1)/2+ N_c (N_t-N_c),
\label{N_3S}
\eeq
respectively. Note that if $N_c$ is not optimized, then $N_c \equiv N_t$, $N_2 = N_t^2$, and the two considered orbital bases
$\{ P_\lambda(r) \}_{\lambda=1}^N$ and $\{ P^+_\lambda(r) \}_{\lambda=1}^N$ yield identical eigenstates
$\{ \psi_{\gamma} \}$ and $\{ \Psi_i^\Gamma \}$.



\subsection{JM method}

This section describes how the required wave functions are constructed, where
the full description of our version of the JM method could be found in \cite{KB10p022708, KFB11}.



The scattering problem is described by
\[ (H - E) | \Psi_E \rangle = 0, \ \ E = e_{\gamma_0} + E_0, \ \ \ E_0 = k_0^2/2, \]
where: $H\equiv H_{a+1}$ (\ref{H_a}); $E$ is the given total energy of the $(a+1)$-electron system;
the target is assumed to be in $\gamma_0$ state before the collision;
$E_0$ is the incident electron energy.
A set of scattering channels $\{ \chi_{\gamma}^{\Gamma} \}$ is defined \cite{BR76p1491,KFB11} by
spin-coupling the target eigenstates $|\psi_{\gamma} \rangle$ and the spin of the scattering electron
$s\equiv \frac{1}{2}$
\beq \barr{l}
| \chi_{\gamma}^{\Gamma} \rangle = \sum_{\mu \mu_\gamma}
C_{S_\gamma \mu_\gamma s \mu}^{S_\Gamma \mu_\Gamma}
|\psi_{\gamma \mu_\gamma} \rangle \ |s \mu  \rangle, \ \ \Gamma \equiv \{S_\Gamma,\mu_\Gamma\},
\earr \label{psi_phi_Gamma}
\eeq
where: $C_{j_1m_1j_2m_2}^{jm} \equiv \langle j_1m_1 j_2 m_2| jm\rangle$ are the Clebsch-Gordan coefficients; $|s \mu \rangle$ is Pauli spinor;
each scattering channel is labeled by a $\{\gamma, \Gamma\}$-pair.
For a given $\Gamma$ and similar to the $a$-electron case (\ref{psi_H_psi}), the $(a+1)$-electron $L^2$ wave functions $\{ \Psi_i^\Gamma \}$
are obtained by diagonalizing $H_{a+1}$ (\ref{H_a}),
where now $\{ \Phi_j^\Gamma \}$ are the Fano's \cite{Fano65,KFB11} subshell-structured $(a+1)$-electron configurations
labeled in some fixed order.


The full scattering wave function $|\Psi_E \rangle$ is approximated by the JM multichannel expansion
\cite{KB10p022708,KFB11},
\beq
\barr{l}
|\Psi_E \rangle \approx \sum_\Gamma | \Psi_N^{\Gamma} \rangle, \\
| \Psi_N^\Gamma \rangle =\sum_j | \Psi_j^\Gamma \rangle \ a_j
+ \sum_{\gamma} \sum_{p=N}^{\infty}
| \Psi_{\gamma p}^{\Gamma} \rangle \ f^{\gamma \gamma_0}_{p}, \\
f^{\gamma \gamma_0}_{p} = (\pi|k_{\gamma}|/2)^{-1/2} \left( s_{p}^{\gamma} \delta_{\gamma \gamma_0}
+ c_{p}^{\gamma} R_{\gamma \gamma_0} \right) ,
\earr \label{Psi_E_N} \eeq
where $k_{\gamma}=\sqrt{2(E-e_{\gamma})}$, and where coefficients $s_{p}^{\gamma}$ and $c_{p}^{\gamma}$
are known \cite{BR76p1491} from the JM method. Let matrices $\hat{k}$ and $\hat{k}^{-1}$ be defined as
\beq
\hat{k}_{\gamma' \gamma} =  \delta_{\gamma \gamma'} |k_{\gamma}|^{1/2}, \ \
\hat{k}_{\gamma' \gamma}^{-1} =  \delta_{\gamma \gamma'} |k_{\gamma}|^{-1/2},
\label{k_matrix} \eeq
then
the reactance matrix $R_{\gamma \gamma_0}$ from \cite{BR76p1491} could be expressed as
\beq \barr{rl}
R &= - \hat{k} Z^{-1}Y\hat{k}^{-1},\\
R_{\gamma \gamma_0} &= - (|k_{\gamma}|/|k_{\gamma_0}|)^{-1/2} \sum_{\gamma'} Z^{-1}_{\gamma \gamma'} Y_{\gamma' \gamma_0},\\
Y_{\gamma' \gamma} &=   W_{\gamma' \gamma} J_{N-1,N}^{\gamma} s_N^{\gamma}
+ \delta_{\gamma \gamma'}s_{N-1}^{\gamma}  ,\\
Z_{\gamma' \gamma} &=  W_{\gamma' \gamma}  J_{N-1,N}^{\gamma} c_N^{\gamma}
+ \delta_{\gamma \gamma'}c_{N-1}^{\gamma}   ,\\
\earr \label{R_matrix} \eeq
where $J_{p'p}^{\gamma} = \langle \xi_{p'} |K - (E-e_\gamma) | \xi_{p}  \rangle$, and
\beq \barr{l}
W_{\gamma' \gamma}= \sum_j  X_j^{\gamma'} X_j^{\gamma} / (E_j-E),
\earr \label{W_gg} \eeq
\beq \barr{l}
X_i^\gamma =  \sum_{\lambda=N_t+1}^N C_{i \gamma\lambda}^\Gamma D_{\lambda, N-1}, \\
C_{i\gamma\lambda}^\Gamma =  \sum_\beta  C^t_{\gamma\beta} C_{i, j=\{\beta \lambda\} } .
\earr \label{X_ig} \eeq



\subsection{Radial electron distribution}
A resonance $(a+1)$-electron state could be further examined using its radial electron distribution (RED).
In the notation of \citet{KFB11} and using (\ref{Psi_from_Fano}),
the probability density of finding an electron at radial distance $r$ is given by
\beq \barr{rl}
f_i(r) &= \sum_\nu \isum \dd {\bf q} \ |\Psi^\Gamma_i({\bf q}, r, \nu)|^2=\\
&= \sum_{jj'} C_{ij}C_{ij'} f_{jj'}(r),\\
\earr \label{f_i} \eeq
\beq
{\bf q} \equiv (q_1, q_2, ..., q_a),
\eeq
\beq
f_{jj'}(r) = \sum_\nu \isum \dd {\bf q} \
\langle \hat{A}\Phi^\Gamma_j({\bf q}, r, \nu) |\hat{A}\Phi^\Gamma_{j'}({\bf q}, r, \nu) \rangle,\\
\label{f_jj} \eeq
where $q_b \equiv (r_b, \nu_b)$ denotes the radial $r_b$ and spin $\nu_b$ coordinates of the $b$th electron.
Before the Fano's formalism \cite{Fano65,KFB11} for the subshell-structured wave-functions could be applied,
Eq.~(\ref{f_jj}) is re-written in terms of one-electron matrix elements as
\beq \barr{l}
f_{jj'}(r) = \frac{1}{a+1} \sum_{b=1}^{a+1}
\langle \hat{A}\Phi^\Gamma_j |  \delta(r-r_b)|\hat{A}\Phi^\Gamma_{j'} \rangle.
\earr \label{f_jj_delta} \eeq
Eq.~(34) of \cite{KFB11} is now directly applicable, arriving at
\beq \barr{rl}
f_{jj'}(r) =&  \frac{1}{a+1} \sum_{\sigma \sigma'} [a_{j \sigma}a_{j' \sigma'}]^{1/2}
 (-1)^{P_\sigma + P_{\sigma'}} \\
& \times \langle \Phi_{j \sigma} | \delta(r-r_{a+1}) |\Phi_{j' \sigma'} \rangle,
\earr \label{RED} \eeq
where $a_{j\lambda}$ is the number of equivalent electrons in the $\lambda$th subshell
of the $j$th $(a+1)$-electron basis function $\Phi_j^\Gamma$.








\section{RESULTS}
\subsection{Basis optimization}
The Laguerre exponential falloff $\lambda_{\rm L}$ is optimized (Fig.~\ref{Fig_LAMBDAS} and Table~\ref{Tab_LAMBDAS})
for each considered pair of $N_c$ and $N_t$ by minimizing
the following maximum positive error (MPE)
\beq
{\rm MPE} = \max_{\gamma = 1}^{n_\gamma} (e_\gamma - e_\gamma^{\mbox{\tiny{DHIF}} }),
\label{MPE}
\eeq
where (i) $n_\gamma$ is the number of first (with the lowest energies) singlet and triplet helium bound states used
in the optimization;
(ii) $\{e_\gamma\}_{\gamma=1}^{n_\gamma}$ are the eigenvalue energies from Eq.~(\ref{psi_H_psi}); and
(iii) $e_\gamma^{\mbox{\tiny{DHIF}}}$ are the corresponding to $e_\gamma$ energies from \citet{DHIF94}, see Table~\ref{Tab_ENGS}.
Fig.~\ref{Fig_LAMBDAS} clearly shows that the helium-ion basis very quickly reaches its descriptive limit, see line labeled $N_c^+=5$.
The limited power of the basis is due to its first $N_c$
orbitals becoming more and more like the $Z=2$ Coulomb field discrete states as $N_t$ increases.
Therefore the $1s$-like state of the "first" electron is not improving even if the best possible $\lambda_{\rm L}$  is chosen
(Table~\ref{Tab_LAMBDAS}).
On the other hand, the Laguerre basis continues to deliver better helium bound states as $N_t$ increases with the same fixed $N_c$,
see the $N_c=5$ line compared with the $N_c^+=5$ line.


Within the JM method, the three-electron matrix elements [Eq.~(\ref{Psi_from_Fano})] are the main computational cost which
grows quadratic with the number of three-electron configurations $N_3$, where
\beq N_3 \sim N_2 N.
\eeq
Therefore the bottom sub-figure in Fig.~\ref{Fig_LAMBDAS} examines the lowest achieved MPE values
against the resulting total two-electron basis size $N_2$.
The result is somewhat surprising in that fixing $N_c$ is in fact produces less optimal basis (for the same $N_2$)
than the unrestricted $N_c=N_t$, subject to $\lambda_{\rm L}$ being independently optimized
for both cases (Table~\ref{Tab_LAMBDAS}).


\subsection{Resonances in $e$-He $S$-wave scattering}

see the resulting energy levels for two $\lambda_{\rm L}$ optimized with $n_\gamma=5$ and $n_\gamma=7$.
Note that $e^{\mbox{\tiny{DHIF}}}(1s2s,^1S)=-2.14418810$ from \cite{DHIF94} is inconsistent
and it is likely an error $-2.14419810$?


TODO Error in exact energy triplet?

[TODO]
Atomic unit of energy (or Hartree) was set to 27.21138386 eV \cite{MTN08}. A tabular
form of the JM and CCC cross sections is available from jmatrix.googlecode.com.


Cross sections are used to verify the negative-ion eigenstates responsible for resonances.
Verification is done by excluding a particular negative-ion state from the JM calculation and observing disappearance of the resonance behavior.
Once identified, their radial electron distributions are used to classify the resonances relative to the target eigenstates.






Again, both CCC and JM methods described the target helium atom
, where the target eigenstates were constructed from the first $N_t$ JM functions (\ref{psi_H_psi}). Convergence in the CCC cross sections
(Figs.~\ref{Fig_He_n2} and \ref{Fig_He_n3}) was achieved at $N_t=?$, where the corresponding JM cross sections
converged at $N_t=?$ and $N=?$.




\begin{table}[htb]
\caption{\label{Tab_LAMBDAS}
The Laguerre exponential falloff $\lambda_{\rm L}$ minimizing the maximum positive error (MPE)
for selected $N_c$ and $N_t$ values, where $n_\gamma=5$ in Eq.~(\ref{MPE}), and $N_2$ is
the total number of helium eigenstates [Eq.~(\ref{N_2})].
All results are for the Laguerre basis [Eq.~(\ref{P_N})] except for the
the "5+" labeled rows obtained with the helium-ion basis
[Eqs.~(\ref{P_ion_sum}), (\ref{P_h_P}) and (\ref{P_ion_N})].
Superscripts indicate powers of 10.
}
\begin{ruledtabular}
%\begin{tabular}{lcr}
\begin{tabular}{llllll}
$N_c$ & $N_t$ & $N_2$ & MPE & $\lambda_{\rm L}$ \\
\hline
5+  & 10  &  75   & 3.4$^{-3}$ &	2.187\\
5+  & 15  & 125   & 3.4$^{-3}$ &	3.177\\
\hline
5  & 10  &  75   & 2.5$^{-3}$ &	1.963 \\
5  & 15  & 125   & 2.9$^{-4}$ &	2.350 \\
5  & 20  & 175   & 1.0$^{-4}$ &	2.960 \\
5  & 25  & 225   & 6.3$^{-5}$ &	3.540 \\
5  & 30  & 275   & 4.3$^{-5}$ &	4.112 \\
5  & 35  & 325   & 3.1$^{-5}$ &	4.681 \\
\hline
10 & 10  & 100  & 3.5$^{-4}$ &	1.728 \\
11 & 11  & 121  & 2.0$^{-4}$ &	1.807 \\
12 & 12  & 144  & 1.2$^{-4}$ &	1.881 \\
13 & 13  & 169  & 8.1$^{-5}$ &	1.964 \\
14 & 14  & 196  & 5.5$^{-5}$ &	2.048 \\
15 & 15  & 225  & 3.8$^{-5}$ &	2.132 \\
16 & 16  & 256  & 2.7$^{-5}$ &	2.220 \\
17 & 17  & 289  & 1.9$^{-5}$ &	2.306 \\
18 & 18  & 324  & 1.4$^{-5}$ &	2.397 \\
19 & 19  & 361  & 1.1$^{-5}$ &	2.488 \\
20 & 20  & 400  & 7.8$^{-6}$ &	2.582 \\
21 & 21  & 441  & 5.8$^{-6}$ &	2.676 \\
22 & 22  & 484  & 4.3$^{-6}$ &	2.773 \\
\end{tabular}
\end{ruledtabular}
\end{table}

\begin{table}[htb]
\caption{\label{Tab_ENGS}
Energies and classifications for $S$-wave helium electron configurations.
Energies $e_i$  and $E_i$ are from Eqs.~(\ref{psi_H_psi}) and (\ref{Psi_from_Fano}), respectively.
$\lambda_{\rm L}=4$, $N_c=N_t$
}

\begin{ruledtabular}
%\begin{tabular}{lcr}
\begin{tabular}{llll}
Classification & threshold & $e_i$  or $E_i$ &    \\
\hline
                      &    & -2.879 028 767 315 &  Ref. \cite{G94}    \\
                      &    & -2.879 028 732   &  Ref. \cite{JB97p2614}    \\
$\mbox{He}(1s^2,^1S)$ &  0 & -2.879 028 569 1 &  $N_t=50$   \\ %-2.879028569120921, -2.1441972587315763, -2.060794037546864,
                      &    & -2.879 028 504   &  $N_t=45$   \\
                      &    & -2.879 027 69    &  Ref. \cite{DHIF94}    \\
                      &    & -2.879 03        &  Ref. \cite{HMR05R}    \\
                      &    & -2.878 95        &  Ref. \cite{BS10p022715}    \\
\hline
$\mbox{He}^-(1s2s,^2S)$  &&		  &  \\
\hline
$\mbox{He}(1s2s,^3S)$    & 0.704 763 712  & -2.174 264 856 2 & $N_t=50$ \\  %-2.174264856287701, -2.0684901366080752,
                         &                & -2.174 264 856 2 & $N_t=45$ \\
                         &                & -2.174 264 80  & \\
                         &                & -2.174 26      & \\
                         &                & -2.174 26      & \\
\hline
$\mbox{He}^-(1s2s,^2S)$  &&      & \\
\hline
$\mbox{He}(1s2s,^1S)$    & 0.734 831 310 & -2.144 197 258 7 &  $N_t=50$ \\
                         &               & -2.144 197 253   &  $N_t=45$   \\
                         &               & -2.144 188 10    &    \\
                         &               & -2.144 20        &    \\
                         &               & -2.144 19        &    \\
\hline
$\mbox{He}^-(1s3s,^2S)$  &&     	 &  \\
\hline
$\mbox{He}(1s3s,^3S)$    & 0.810 538 432 & -2.068 490 136 6 & $N_t=50$  \\
                         &               & -2.068 490 135   & $N_t=45$\\
                         &               & -2.068 490 12    & \\
                         &               & -2.068 49        & \\
                         &               & -2.068 48        & \\
\hline
$\mbox{He}^-(1s3s,^2S)$  &&     	 &  \\
\hline
$\mbox{He}(1s3s,^1S)$    & 0.818 234 531 & -2.060 794 037 5 & $N_t=50$ \\
                         &               & -2.060 794 025   & $N_t=45$\\
                         &               & -2.060 788 24    & \\
                         &               & -2.060 79        & \\
                         &               & -2.060 79        & \\
\hline
$\mbox{He}^-(1s3s,^2S)$  &&       &	   \\
\hline
$\mbox{He}(1s4s,^3S)$    & 0.842 589 989 & -2.036 438 58    & Ref. \cite{DHIF94} \\
\hline
$\mbox{He}^+(1s)$        & 0.879 028 569 & -2 	 &    \\
\end{tabular}
\end{ruledtabular}
\end{table}


\section{CONCLUSIONS}


In order to achieve the stated goal we apply the $J$-matrix (JM) approach to electron-atom scattering,
which has been recently revised by merging it with the Fano's multi-configuration interaction matrix elements \cite{Fano65}.
In that preceding JM paper \cite{BF11},
the $S$-wave $e$-He scattering problem was solved within the frozen-core (FC) model of helium for
the elastic, $2^{1,3}S$-excitation, and single ionization cross sections for impact energies in the range 0.1-1000eV.
The reported in \cite{BF11} "proof-of-principle" JM calculations were in complete agreement with the convergent-close-coupling (CCC) method,
within the FC model.
In this sequel, the scattering target helium atom is described at much higher level of accuracy overcoming the FC model.
It is found that the theory in \cite{BF11} is sufficient to fully solve the $S$-model below the single ionization threshold.
The presented JM results (1-30 eV) are confirmed by the corresponding CCC calculations providing
total elastic, $2^{1,3}S$ and $3^{1,3}S$ excitation cross sections with a "benchmark"-level of accuracy for the first time for the considered $S$-wave model.


\begin{acknowledgments}
This work was supported by the Australian Research Council. IB
acknowledges the Australian National Computational Infrastructure
Facility and its Western Australian node iVEC.
\end{acknowledgments}



\bibliographystyle{apsrev}
\bibliography{../bibtex/qm_references}

\end{document}
